If you cannot find the answer to your question here, and you have read \hyperlink{V1__7__Primer_8md}{Primer} and \hyperlink{V1__7__AdvancedGuide_8md}{Advanced\+Guide}, send it to \href{mailto:googletestframework@googlegroups.com}{\tt googletestframework@googlegroups.\+com}.

\subsection*{Why should I use Google Test instead of my favorite C++ testing framework?}

First, let us say clearly that we don\textquotesingle{}t want to get into the debate of which C++ testing framework is {\bfseries the best}. There exist many fine frameworks for writing C++ tests, and we have tremendous respect for the developers and users of them. We don\textquotesingle{}t think there is (or will be) a single best framework -\/ you have to pick the right tool for the particular task you are tackling.

We created Google Test because we couldn\textquotesingle{}t find the right combination of features and conveniences in an existing framework to satisfy {\itshape our} needs. The following is a list of things that {\itshape we} like about Google Test. We don\textquotesingle{}t claim them to be unique to Google Test -\/ rather, the combination of them makes Google Test the choice for us. We hope this list can help you decide whether it is for you too.


\begin{DoxyItemize}
\item Google Test is designed to be portable\+: it doesn\textquotesingle{}t require exceptions or R\+T\+TI; it works around various bugs in various compilers and environments; etc. As a result, it works on Linux, Mac OS X, Windows and several embedded operating systems.
\item Nonfatal assertions ({\ttfamily E\+X\+P\+E\+C\+T\+\_\+$\ast$}) have proven to be great time savers, as they allow a test to report multiple failures in a single edit-\/compile-\/test cycle.
\item It\textquotesingle{}s easy to write assertions that generate informative messages\+: you just use the stream syntax to append any additional information, e.\+g. {\ttfamily \hyperlink{gtest_8h_a1a6db8b1338ee7040329322b77779086}{A\+S\+S\+E\+R\+T\+\_\+\+E\+Q(5, Foo(i))} $<$$<$ \char`\"{} where i = \char`\"{} $<$$<$ i;}. It doesn\textquotesingle{}t require a new set of macros or special functions.
\item Google Test automatically detects your tests and doesn\textquotesingle{}t require you to enumerate them in order to run them.
\item Death tests are pretty handy for ensuring that your asserts in production code are triggered by the right conditions.
\item {\ttfamily S\+C\+O\+P\+E\+D\+\_\+\+T\+R\+A\+CE} helps you understand the context of an assertion failure when it comes from inside a sub-\/routine or loop.
\item You can decide which tests to run using name patterns. This saves time when you want to quickly reproduce a test failure.
\item Google Test can generate X\+ML test result reports that can be parsed by popular continuous build system like Hudson.
\item Simple things are easy in Google Test, while hard things are possible\+: in addition to advanced features like \href{V1_7_AdvancedGuide.md#global-set-up-and-tear-down}{\tt global test environments} and tests parameterized by \href{V1_7_AdvancedGuide.md#value-parameterized-tests}{\tt values} or \href{V1_7_AdvancedGuide.md#typed-tests}{\tt types}, Google Test supports various ways for the user to extend the framework -- if Google Test doesn\textquotesingle{}t do something out of the box, chances are that a user can implement the feature using Google Test\textquotesingle{}s public A\+PI, without changing Google Test itself. In particular, you can\+:
\begin{DoxyItemize}
\item expand your testing vocabulary by defining \href{V1_7_AdvancedGuide.md#predicate-assertions-for-better-error-messages}{\tt custom predicates},
\item teach Google Test how to \href{V1_7_AdvancedGuide.md#teaching-google-test-how-to-print-your-values}{\tt print your types},
\item define your own testing macros or utilities and verify them using Google Test\textquotesingle{}s \href{V1_7_AdvancedGuide.md#catching-failures}{\tt Service Provider Interface}, and
\item reflect on the test cases or change the test output format by intercepting the \href{V1_7_AdvancedGuide.md#extending-google-test-by-handling-test-events}{\tt test events}.
\end{DoxyItemize}
\end{DoxyItemize}

\subsection*{I\textquotesingle{}m getting warnings when compiling Google Test. Would you fix them?}

We strive to minimize compiler warnings Google Test generates. Before releasing a new version, we test to make sure that it doesn\textquotesingle{}t generate warnings when compiled using its C\+Make script on Windows, Linux, and Mac OS.

Unfortunately, this doesn\textquotesingle{}t mean you are guaranteed to see no warnings when compiling Google Test in your environment\+:


\begin{DoxyItemize}
\item You may be using a different compiler as we use, or a different version of the same compiler. We cannot possibly test for all compilers.
\item You may be compiling on a different platform as we do.
\item Your project may be using different compiler flags as we do.
\end{DoxyItemize}

It is not always possible to make Google Test warning-\/free for everyone. Or, it may not be desirable if the warning is rarely enabled and fixing the violations makes the code more complex.

If you see warnings when compiling Google Test, we suggest that you use the {\ttfamily -\/isystem} flag (assuming your are using G\+CC) to mark Google Test headers as system headers. That\textquotesingle{}ll suppress warnings from Google Test headers.

\subsection*{Why should not test case names and test names contain underscore?}

Underscore ({\ttfamily \+\_\+}) is special, as C++ reserves the following to be used by the compiler and the standard library\+:


\begin{DoxyEnumerate}
\item any identifier that starts with an {\ttfamily \+\_\+} followed by an upper-\/case letter, and
\end{DoxyEnumerate}
\begin{DoxyEnumerate}
\item any identifier that containers two consecutive underscores (i.\+e. {\ttfamily \+\_\+\+\_\+}) {\itshape anywhere} in its name.
\end{DoxyEnumerate}

User code is {\itshape prohibited} from using such identifiers.

Now let\textquotesingle{}s look at what this means for {\ttfamily T\+E\+ST} and {\ttfamily T\+E\+S\+T\+\_\+F}.

Currently {\ttfamily \hyperlink{gtest_8h_ad8b332753515c0ab8baada563c2547eb}{T\+E\+S\+T(\+Test\+Case\+Name, Test\+Name)}} generates a class named {\ttfamily Test\+Case\+Name\+\_\+\+Test\+Name\+\_\+\+Test}. What happens if {\ttfamily Test\+Case\+Name} or {\ttfamily Test\+Name} contains {\ttfamily \+\_\+}?


\begin{DoxyEnumerate}
\item If {\ttfamily Test\+Case\+Name} starts with an {\ttfamily \+\_\+} followed by an upper-\/case letter (say, {\ttfamily \+\_\+\+Foo}), we end up with {\ttfamily \+\_\+\+Foo\+\_\+\+Test\+Name\+\_\+\+Test}, which is reserved and thus invalid.
\end{DoxyEnumerate}
\begin{DoxyEnumerate}
\item If {\ttfamily Test\+Case\+Name} ends with an {\ttfamily \+\_\+} (say, {\ttfamily Foo\+\_\+}), we get {\ttfamily Foo\+\_\+\+\_\+\+Test\+Name\+\_\+\+Test}, which is invalid.
\end{DoxyEnumerate}
\begin{DoxyEnumerate}
\item If {\ttfamily Test\+Name} starts with an {\ttfamily \+\_\+} (say, {\ttfamily \+\_\+\+Bar}), we get {\ttfamily Test\+Case\+Name\+\_\+\+\_\+\+Bar\+\_\+\+Test}, which is invalid.
\end{DoxyEnumerate}
\begin{DoxyEnumerate}
\item If {\ttfamily Test\+Name} ends with an {\ttfamily \+\_\+} (say, {\ttfamily Bar\+\_\+}), we get {\ttfamily Test\+Case\+Name\+\_\+\+Bar\+\_\+\+\_\+\+Test}, which is invalid.
\end{DoxyEnumerate}

So clearly {\ttfamily Test\+Case\+Name} and {\ttfamily Test\+Name} cannot start or end with {\ttfamily \+\_\+} (Actually, {\ttfamily Test\+Case\+Name} can start with {\ttfamily \+\_\+} -- as long as the {\ttfamily \+\_\+} isn\textquotesingle{}t followed by an upper-\/case letter. But that\textquotesingle{}s getting complicated. So for simplicity we just say that it cannot start with {\ttfamily \+\_\+}.).

It may seem fine for {\ttfamily Test\+Case\+Name} and {\ttfamily Test\+Name} to contain {\ttfamily \+\_\+} in the middle. However, consider this\+: 
\begin{DoxyCode}
1 TEST(Time, Flies\_Like\_An\_Arrow) \{ ... \}
2 TEST(Time\_Flies, Like\_An\_Arrow) \{ ... \}
\end{DoxyCode}


Now, the two {\ttfamily T\+E\+ST}s will both generate the same class ({\ttfamily Time\+\_\+\+Files\+\_\+\+Like\+\_\+\+An\+\_\+\+Arrow\+\_\+\+Test}). That\textquotesingle{}s not good.

So for simplicity, we just ask the users to avoid {\ttfamily \+\_\+} in {\ttfamily Test\+Case\+Name} and {\ttfamily Test\+Name}. The rule is more constraining than necessary, but it\textquotesingle{}s simple and easy to remember. It also gives Google Test some wiggle room in case its implementation needs to change in the future.

If you violate the rule, there may not be immediately consequences, but your test may (just may) break with a new compiler (or a new version of the compiler you are using) or with a new version of Google Test. Therefore it\textquotesingle{}s best to follow the rule.

\subsection*{Why is it not recommended to install a pre-\/compiled copy of Google Test (for example, into /usr/local)?}

In the early days, we said that you could install compiled Google Test libraries on {\ttfamily $\ast$}nix systems using {\ttfamily make install}. Then every user of your machine can write tests without recompiling Google Test.

This seemed like a good idea, but it has a got-\/cha\+: every user needs to compile his tests using the {\itshape same} compiler flags used to compile the installed Google Test libraries; otherwise he may run into undefined behaviors (i.\+e. the tests can behave strangely and may even crash for no obvious reasons).

Why? Because C++ has this thing called the One-\/\+Definition Rule\+: if two C++ source files contain different definitions of the same class/function/variable, and you link them together, you violate the rule. The linker may or may not catch the error (in many cases it\textquotesingle{}s not required by the C++ standard to catch the violation). If it doesn\textquotesingle{}t, you get strange run-\/time behaviors that are unexpected and hard to debug.

If you compile Google Test and your test code using different compiler flags, they may see different definitions of the same class/function/variable (e.\+g. due to the use of {\ttfamily \#if} in Google Test). Therefore, for your sanity, we recommend to avoid installing pre-\/compiled Google Test libraries. Instead, each project should compile Google Test itself such that it can be sure that the same flags are used for both Google Test and the tests.

\subsection*{How do I generate 64-\/bit binaries on Windows (using Visual Studio 2008)?}

(Answered by Trevor Robinson)

Load the supplied Visual Studio solution file, either {\ttfamily msvc\textbackslash{}gtest-\/md.\+sln} or {\ttfamily msvc\textbackslash{}gtest.\+sln}. Go through the migration wizard to migrate the solution and project files to Visual Studio 2008. Select {\ttfamily Configuration Manager...} from the {\ttfamily Build} menu. Select {\ttfamily $<$New...$>$} from the {\ttfamily Active solution platform} dropdown. Select {\ttfamily x64} from the new platform dropdown, leave {\ttfamily Copy settings from} set to {\ttfamily Win32} and {\ttfamily Create new project platforms} checked, then click {\ttfamily OK}. You now have {\ttfamily Win32} and {\ttfamily x64} platform configurations, selectable from the {\ttfamily Standard} toolbar, which allow you to toggle between building 32-\/bit or 64-\/bit binaries (or both at once using Batch Build).

In order to prevent build output files from overwriting one another, you\textquotesingle{}ll need to change the {\ttfamily Intermediate Directory} settings for the newly created platform configuration across all the projects. To do this, multi-\/select (e.\+g. using shift-\/click) all projects (but not the solution) in the {\ttfamily Solution Explorer}. Right-\/click one of them and select {\ttfamily Properties}. In the left pane, select {\ttfamily Configuration Properties}, and from the {\ttfamily Configuration} dropdown, select {\ttfamily All Configurations}. Make sure the selected platform is {\ttfamily x64}. For the {\ttfamily Intermediate Directory} setting, change the value from {\ttfamily \textbackslash{}} to {\ttfamily \textbackslash{}}. Click {\ttfamily OK} and then build the solution. When the build is complete, the 64-\/bit binaries will be in the {\ttfamily msvc\textbackslash{}x64\textbackslash{}Debug} directory.

\subsection*{Can I use Google Test on Min\+GW?}

We haven\textquotesingle{}t tested this ourselves, but Per Abrahamsen reported that he was able to compile and install Google Test successfully when using Min\+GW from Cygwin. You\textquotesingle{}ll need to configure it with\+:

{\ttfamily P\+A\+T\+H/\+T\+O/configure CC=\char`\"{}gcc -\/mno-\/cygwin\char`\"{} C\+XX=\char`\"{}g++ -\/mno-\/cygwin\char`\"{}}

You should be able to replace the {\ttfamily -\/mno-\/cygwin} option with direct links to the real Min\+GW binaries, but we haven\textquotesingle{}t tried that.

Caveats\+:


\begin{DoxyItemize}
\item There are many warnings when compiling.
\item {\ttfamily make check} will produce some errors as not all tests for Google Test itself are compatible with Min\+GW.
\end{DoxyItemize}

We also have reports on successful cross compilation of Google Test Min\+GW binaries on Linux using \href{http://wiki.wxwidgets.org/Cross-Compiling_Under_Linux#Cross-compiling_under_Linux_for_MS_Windows}{\tt these instructions} on the Wx\+Widgets site.

Please contact {\ttfamily googletestframework@googlegroups.\+com} if you are interested in improving the support for Min\+GW.

\subsection*{Why does Google Test support \hyperlink{gtest_8h_a4159019abda84f5366acdb7604ff220a}{E\+X\+P\+E\+C\+T\+\_\+\+E\+Q(\+N\+U\+L\+L, ptr)} and \hyperlink{gtest_8h_a1a6db8b1338ee7040329322b77779086}{A\+S\+S\+E\+R\+T\+\_\+\+E\+Q(\+N\+U\+L\+L, ptr)} but not \hyperlink{gtest_8h_a6ae7443947f25abc58bfcfcfc56b0d75}{E\+X\+P\+E\+C\+T\+\_\+\+N\+E(\+N\+U\+L\+L, ptr)} and \hyperlink{gtest_8h_aa866c8dece57912e6f51495ed3e8d8d5}{A\+S\+S\+E\+R\+T\+\_\+\+N\+E(\+N\+U\+L\+L, ptr)}?}

Due to some peculiarity of C++, it requires some non-\/trivial template meta programming tricks to support using {\ttfamily N\+U\+LL} as an argument of the {\ttfamily E\+X\+P\+E\+C\+T\+\_\+\+X\+X()} and {\ttfamily A\+S\+S\+E\+R\+T\+\_\+\+X\+X()} macros. Therefore we only do it where it\textquotesingle{}s most needed (otherwise we make the implementation of Google Test harder to maintain and more error-\/prone than necessary).

The {\ttfamily \hyperlink{gtest_8h_a4159019abda84f5366acdb7604ff220a}{E\+X\+P\+E\+C\+T\+\_\+\+E\+Q()}} macro takes the {\itshape expected} value as its first argument and the {\itshape actual} value as the second. It\textquotesingle{}s reasonable that someone wants to write {\ttfamily \hyperlink{gtest_8h_a4159019abda84f5366acdb7604ff220a}{E\+X\+P\+E\+C\+T\+\_\+\+E\+Q(\+N\+U\+L\+L, some\+\_\+expression)}}, and this indeed was requested several times. Therefore we implemented it.

The need for {\ttfamily \hyperlink{gtest_8h_a6ae7443947f25abc58bfcfcfc56b0d75}{E\+X\+P\+E\+C\+T\+\_\+\+N\+E(\+N\+U\+L\+L, ptr)}} isn\textquotesingle{}t nearly as strong. When the assertion fails, you already know that {\ttfamily ptr} must be {\ttfamily N\+U\+LL}, so it doesn\textquotesingle{}t add any information to print ptr in this case. That means {\ttfamily E\+X\+P\+E\+C\+T\+\_\+\+T\+R\+UE(ptr != N\+U\+LL)} works just as well.

If we were to support {\ttfamily \hyperlink{gtest_8h_a6ae7443947f25abc58bfcfcfc56b0d75}{E\+X\+P\+E\+C\+T\+\_\+\+N\+E(\+N\+U\+L\+L, ptr)}}, for consistency we\textquotesingle{}ll have to support {\ttfamily \hyperlink{gtest_8h_a6ae7443947f25abc58bfcfcfc56b0d75}{E\+X\+P\+E\+C\+T\+\_\+\+N\+E(ptr, N\+U\+L\+L)}} as well, as unlike {\ttfamily E\+X\+P\+E\+C\+T\+\_\+\+EQ}, we don\textquotesingle{}t have a convention on the order of the two arguments for {\ttfamily E\+X\+P\+E\+C\+T\+\_\+\+NE}. This means using the template meta programming tricks twice in the implementation, making it even harder to understand and maintain. We believe the benefit doesn\textquotesingle{}t justify the cost.

Finally, with the growth of Google \hyperlink{classMock}{Mock}\textquotesingle{}s \href{../../CookBook.md#using-matchers-in-google-test-assertions}{\tt matcher} library, we are encouraging people to use the unified {\ttfamily \hyperlink{gmock-matchers_8h_ac31e206123aa702e1152bb2735b31409}{E\+X\+P\+E\+C\+T\+\_\+\+T\+H\+A\+T(value, matcher)}} syntax more often in tests. One significant advantage of the matcher approach is that matchers can be easily combined to form new matchers, while the {\ttfamily E\+X\+P\+E\+C\+T\+\_\+\+NE}, etc, macros cannot be easily combined. Therefore we want to invest more in the matchers than in the {\ttfamily E\+X\+P\+E\+C\+T\+\_\+\+X\+X()} macros.

\subsection*{Does Google Test support running tests in parallel?}

Test runners tend to be tightly coupled with the build/test environment, and Google Test doesn\textquotesingle{}t try to solve the problem of running tests in parallel. Instead, we tried to make Google Test work nicely with test runners. For example, Google Test\textquotesingle{}s X\+ML report contains the time spent on each test, and its {\ttfamily gtest\+\_\+list\+\_\+tests} and {\ttfamily gtest\+\_\+filter} flags can be used for splitting the execution of test methods into multiple processes. These functionalities can help the test runner run the tests in parallel.

\subsection*{Why don\textquotesingle{}t Google Test run the tests in different threads to speed things up?}

It\textquotesingle{}s difficult to write thread-\/safe code. Most tests are not written with thread-\/safety in mind, and thus may not work correctly in a multi-\/threaded setting.

If you think about it, it\textquotesingle{}s already hard to make your code work when you know what other threads are doing. It\textquotesingle{}s much harder, and sometimes even impossible, to make your code work when you don\textquotesingle{}t know what other threads are doing (remember that test methods can be added, deleted, or modified after your test was written). If you want to run the tests in parallel, you\textquotesingle{}d better run them in different processes.

\subsection*{Why aren\textquotesingle{}t Google Test assertions implemented using exceptions?}

Our original motivation was to be able to use Google Test in projects that disable exceptions. Later we realized some additional benefits of this approach\+:


\begin{DoxyEnumerate}
\item Throwing in a destructor is undefined behavior in C++. Not using exceptions means Google Test\textquotesingle{}s assertions are safe to use in destructors.
\end{DoxyEnumerate}
\begin{DoxyEnumerate}
\item The {\ttfamily E\+X\+P\+E\+C\+T\+\_\+$\ast$} family of macros will continue even after a failure, allowing multiple failures in a {\ttfamily T\+E\+ST} to be reported in a single run. This is a popular feature, as in C++ the edit-\/compile-\/test cycle is usually quite long and being able to fixing more than one thing at a time is a blessing.
\end{DoxyEnumerate}
\begin{DoxyEnumerate}
\item If assertions are implemented using exceptions, a test may falsely ignore a failure if it\textquotesingle{}s caught by user code\+: 
\begin{DoxyCode}
1 try \{ ... ASSERT\_TRUE(...) ... \}
2 catch (...) \{ ... \}
\end{DoxyCode}
 The above code will pass even if the {\ttfamily A\+S\+S\+E\+R\+T\+\_\+\+T\+R\+UE} throws. While it\textquotesingle{}s unlikely for someone to write this in a test, it\textquotesingle{}s possible to run into this pattern when you write assertions in callbacks that are called by the code under test.
\end{DoxyEnumerate}

The downside of not using exceptions is that {\ttfamily A\+S\+S\+E\+R\+T\+\_\+$\ast$} (implemented using {\ttfamily return}) will only abort the current function, not the current {\ttfamily T\+E\+ST}.

\subsection*{Why do we use two different macros for tests with and without fixtures?}

Unfortunately, C++\textquotesingle{}s macro system doesn\textquotesingle{}t allow us to use the same macro for both cases. One possibility is to provide only one macro for tests with fixtures, and require the user to define an empty fixture sometimes\+:


\begin{DoxyCode}
1 class FooTest : public ::testing::Test \{\};
2 
3 TEST\_F(FooTest, DoesThis) \{ ... \}
\end{DoxyCode}
 or 
\begin{DoxyCode}
1 typedef ::testing::Test FooTest;
2 
3 TEST\_F(FooTest, DoesThat) \{ ... \}
\end{DoxyCode}


Yet, many people think this is one line too many. \+:-\/) Our goal was to make it really easy to write tests, so we tried to make simple tests trivial to create. That means using a separate macro for such tests.

We think neither approach is ideal, yet either of them is reasonable. In the end, it probably doesn\textquotesingle{}t matter much either way.

\subsection*{Why don\textquotesingle{}t we use structs as test fixtures?}

We like to use structs only when representing passive data. This distinction between structs and classes is good for documenting the intent of the code\textquotesingle{}s author. Since test fixtures have logic like {\ttfamily Set\+Up()} and {\ttfamily Tear\+Down()}, they are better defined as classes.

\subsection*{Why are death tests implemented as assertions instead of using a test runner?}

Our goal was to make death tests as convenient for a user as C++ possibly allows. In particular\+:


\begin{DoxyItemize}
\item The runner-\/style requires to split the information into two pieces\+: the definition of the death test itself, and the specification for the runner on how to run the death test and what to expect. The death test would be written in C++, while the runner spec may or may not be. A user needs to carefully keep the two in sync. {\ttfamily A\+S\+S\+E\+R\+T\+\_\+\+D\+E\+A\+T\+H(statement, expected\+\_\+message)} specifies all necessary information in one place, in one language, without boilerplate code. It is very declarative.
\item {\ttfamily A\+S\+S\+E\+R\+T\+\_\+\+D\+E\+A\+TH} has a similar syntax and error-\/reporting semantics as other Google Test assertions, and thus is easy to learn.
\item {\ttfamily A\+S\+S\+E\+R\+T\+\_\+\+D\+E\+A\+TH} can be mixed with other assertions and other logic at your will. You are not limited to one death test per test method. For example, you can write something like\+: 
\begin{DoxyCode}
1 if (FooCondition()) \{
2   ASSERT\_DEATH(Bar(), "blah");
3 \} else \{
4   ASSERT\_EQ(5, Bar());
5 \}
\end{DoxyCode}
 If you prefer one death test per test method, you can write your tests in that style too, but we don\textquotesingle{}t want to impose that on the users. The fewer artificial limitations the better.
\item {\ttfamily A\+S\+S\+E\+R\+T\+\_\+\+D\+E\+A\+TH} can reference local variables in the current function, and you can decide how many death tests you want based on run-\/time information. For example, 
\begin{DoxyCode}
1 const int count = GetCount();  // Only known at run time.
2 for (int i = 1; i <= count; i++) \{
3   ASSERT\_DEATH(\{
4     double* buffer = new double[i];
5     ... initializes buffer ...
6     Foo(buffer, i)
7   \}, "blah blah");
8 \}
\end{DoxyCode}
 The runner-\/based approach tends to be more static and less flexible, or requires more user effort to get this kind of flexibility.
\end{DoxyItemize}

Another interesting thing about {\ttfamily A\+S\+S\+E\+R\+T\+\_\+\+D\+E\+A\+TH} is that it calls {\ttfamily fork()} to create a child process to run the death test. This is lightening fast, as {\ttfamily fork()} uses copy-\/on-\/write pages and incurs almost zero overhead, and the child process starts from the user-\/supplied statement directly, skipping all global and local initialization and any code leading to the given statement. If you launch the child process from scratch, it can take seconds just to load everything and start running if the test links to many libraries dynamically.

\subsection*{My death test modifies some state, but the change seems lost after the death test finishes. Why?}

Death tests ({\ttfamily E\+X\+P\+E\+C\+T\+\_\+\+D\+E\+A\+TH}, etc) are executed in a sub-\/process s.\+t. the expected crash won\textquotesingle{}t kill the test program (i.\+e. the parent process). As a result, any in-\/memory side effects they incur are observable in their respective sub-\/processes, but not in the parent process. You can think of them as running in a parallel universe, more or less.

\subsection*{The compiler complains about \char`\"{}undefined references\char`\"{} to some static const member variables, but I did define them in the class body. What\textquotesingle{}s wrong?}

If your class has a static data member\+:


\begin{DoxyCode}
1 // foo.h
2 class Foo \{
3   ...
4   static const int kBar = 100;
5 \};
\end{DoxyCode}


You also need to define it {\itshape outside} of the class body in {\ttfamily foo.\+cc}\+:


\begin{DoxyCode}
1 const int Foo::kBar;  // No initializer here.
\end{DoxyCode}


Otherwise your code is $\ast$$\ast$invalid C++$\ast$$\ast$, and may break in unexpected ways. In particular, using it in Google Test comparison assertions ({\ttfamily E\+X\+P\+E\+C\+T\+\_\+\+EQ}, etc) will generate an \char`\"{}undefined reference\char`\"{} linker error.

\subsection*{I have an interface that has several implementations. Can I write a set of tests once and repeat them over all the implementations?}

Google Test doesn\textquotesingle{}t yet have good support for this kind of tests, or data-\/driven tests in general. We hope to be able to make improvements in this area soon.

\subsection*{Can I derive a test fixture from another?}

Yes.

Each test fixture has a corresponding and same named test case. This means only one test case can use a particular fixture. Sometimes, however, multiple test cases may want to use the same or slightly different fixtures. For example, you may want to make sure that all of a G\+UI library\textquotesingle{}s test cases don\textquotesingle{}t leak important system resources like fonts and brushes.

In Google Test, you share a fixture among test cases by putting the shared logic in a base test fixture, then deriving from that base a separate fixture for each test case that wants to use this common logic. You then use {\ttfamily \hyperlink{gtest_8h_a0ee66d464d1a06c20c1929cae09d8758}{T\+E\+S\+T\+\_\+\+F()}} to write tests using each derived fixture.

Typically, your code looks like this\+:


\begin{DoxyCode}
1 // Defines a base test fixture.
2 class BaseTest : public ::testing::Test \{
3   protected:
4    ...
5 \};
6 
7 // Derives a fixture FooTest from BaseTest.
8 class FooTest : public BaseTest \{
9   protected:
10     virtual void SetUp() \{
11       BaseTest::SetUp();  // Sets up the base fixture first.
12       ... additional set-up work ...
13     \}
14     virtual void TearDown() \{
15       ... clean-up work for FooTest ...
16       BaseTest::TearDown();  // Remember to tear down the base fixture
17                              // after cleaning up FooTest!
18     \}
19     ... functions and variables for FooTest ...
20 \};
21 
22 // Tests that use the fixture FooTest.
23 TEST\_F(FooTest, Bar) \{ ... \}
24 TEST\_F(FooTest, Baz) \{ ... \}
25 
26 ... additional fixtures derived from BaseTest ...
\end{DoxyCode}


If necessary, you can continue to derive test fixtures from a derived fixture. Google Test has no limit on how deep the hierarchy can be.

For a complete example using derived test fixtures, see \href{../samples/sample5_unittest.cc}{\tt sample5}.

\subsection*{My compiler complains \char`\"{}void value not ignored as it ought to be.\char`\"{} What does this mean?}

You\textquotesingle{}re probably using an {\ttfamily A\+S\+S\+E\+R\+T\+\_\+$\ast$()} in a function that doesn\textquotesingle{}t return {\ttfamily void}. {\ttfamily A\+S\+S\+E\+R\+T\+\_\+$\ast$()} can only be used in {\ttfamily void} functions.

\subsection*{My death test hangs (or seg-\/faults). How do I fix it?}

In Google Test, death tests are run in a child process and the way they work is delicate. To write death tests you really need to understand how they work. Please make sure you have read this.

In particular, death tests don\textquotesingle{}t like having multiple threads in the parent process. So the first thing you can try is to eliminate creating threads outside of {\ttfamily E\+X\+P\+E\+C\+T\+\_\+\+D\+E\+A\+T\+H()}.

Sometimes this is impossible as some library you must use may be creating threads before {\ttfamily \hyperlink{app_2main_8cpp_ae66f6b31b5ad750f1fe042a706a4e3d4}{main()}} is even reached. In this case, you can try to minimize the chance of conflicts by either moving as many activities as possible inside {\ttfamily E\+X\+P\+E\+C\+T\+\_\+\+D\+E\+A\+T\+H()} (in the extreme case, you want to move everything inside), or leaving as few things as possible in it. Also, you can try to set the death test style to {\ttfamily \char`\"{}threadsafe\char`\"{}}, which is safer but slower, and see if it helps.

If you go with thread-\/safe death tests, remember that they rerun the test program from the beginning in the child process. Therefore make sure your program can run side-\/by-\/side with itself and is deterministic.

In the end, this boils down to good concurrent programming. You have to make sure that there is no race conditions or dead locks in your program. No silver bullet -\/ sorry!

\subsection*{Should I use the constructor/destructor of the test fixture or the set-\/up/tear-\/down function?}

The first thing to remember is that Google Test does not reuse the same test fixture object across multiple tests. For each {\ttfamily T\+E\+S\+T\+\_\+F}, Google Test will create a fresh test fixture object, {\itshape immediately} call {\ttfamily Set\+Up()}, run the test, call {\ttfamily Tear\+Down()}, and then {\itshape immediately} delete the test fixture object. Therefore, there is no need to write a {\ttfamily Set\+Up()} or {\ttfamily Tear\+Down()} function if the constructor or destructor already does the job.

You may still want to use {\ttfamily Set\+Up()/\+Tear\+Down()} in the following cases\+:
\begin{DoxyItemize}
\item If the tear-\/down operation could throw an exception, you must use {\ttfamily Tear\+Down()} as opposed to the destructor, as throwing in a destructor leads to undefined behavior and usually will kill your program right away. Note that many standard libraries (like S\+TL) may throw when exceptions are enabled in the compiler. Therefore you should prefer {\ttfamily Tear\+Down()} if you want to write portable tests that work with or without exceptions.
\item The assertion macros throw an exception when flag {\ttfamily -\/-\/gtest\+\_\+throw\+\_\+on\+\_\+failure} is specified. Therefore, you shouldn\textquotesingle{}t use Google Test assertions in a destructor if you plan to run your tests with this flag.
\item In a constructor or destructor, you cannot make a virtual function call on this object. (You can call a method declared as virtual, but it will be statically bound.) Therefore, if you need to call a method that will be overriden in a derived class, you have to use {\ttfamily Set\+Up()/\+Tear\+Down()}.
\end{DoxyItemize}

\subsection*{The compiler complains \char`\"{}no matching function to call\char`\"{} when I use A\+S\+S\+E\+R\+T\+\_\+\+P\+R\+E\+Dn. How do I fix it?}

If the predicate function you use in {\ttfamily A\+S\+S\+E\+R\+T\+\_\+\+P\+R\+E\+D$\ast$} or {\ttfamily E\+X\+P\+E\+C\+T\+\_\+\+P\+R\+E\+D$\ast$} is overloaded or a template, the compiler will have trouble figuring out which overloaded version it should use. {\ttfamily A\+S\+S\+E\+R\+T\+\_\+\+P\+R\+E\+D\+\_\+\+F\+O\+R\+M\+A\+T$\ast$} and {\ttfamily E\+X\+P\+E\+C\+T\+\_\+\+P\+R\+E\+D\+\_\+\+F\+O\+R\+M\+A\+T$\ast$} don\textquotesingle{}t have this problem.

If you see this error, you might want to switch to {\ttfamily (A\+S\+S\+E\+R\+T$\vert$\+E\+X\+P\+E\+CT)\+\_\+\+P\+R\+E\+D\+\_\+\+F\+O\+R\+M\+A\+T$\ast$}, which will also give you a better failure message. If, however, that is not an option, you can resolve the problem by explicitly telling the compiler which version to pick.

For example, suppose you have


\begin{DoxyCode}
1 bool IsPositive(int n) \{
2   return n > 0;
3 \}
4 bool IsPositive(double x) \{
5   return x > 0;
6 \}
\end{DoxyCode}


you will get a compiler error if you write


\begin{DoxyCode}
1 EXPECT\_PRED1(IsPositive, 5);
\end{DoxyCode}


However, this will work\+:


\begin{DoxyCode}
1 EXPECT\_PRED1(*static\_cast<bool (*)(int)>*(IsPositive), 5);
\end{DoxyCode}


(The stuff inside the angled brackets for the {\ttfamily static\+\_\+cast} operator is the type of the function pointer for the {\ttfamily int}-\/version of {\ttfamily \hyperlink{namespacetesting_1_1gmock__matchers__test_a70e728cf67d0224c3ebb9eb8959cc39d}{Is\+Positive()}}.)

As another example, when you have a template function


\begin{DoxyCode}
1 template <typename T>
2 bool IsNegative(T x) \{
3   return x < 0;
4 \}
\end{DoxyCode}


you can use it in a predicate assertion like this\+:


\begin{DoxyCode}
1 ASSERT\_PRED1(IsNegative*<int>*, -5);
\end{DoxyCode}


Things are more interesting if your template has more than one parameters. The following won\textquotesingle{}t compile\+:


\begin{DoxyCode}
1 ASSERT\_PRED2(*GreaterThan<int, int>*, 5, 0);
\end{DoxyCode}


as the C++ pre-\/processor thinks you are giving {\ttfamily A\+S\+S\+E\+R\+T\+\_\+\+P\+R\+E\+D2} 4 arguments, which is one more than expected. The workaround is to wrap the predicate function in parentheses\+:


\begin{DoxyCode}
1 ASSERT\_PRED2(*(GreaterThan<int, int>)*, 5, 0);
\end{DoxyCode}


\subsection*{My compiler complains about \char`\"{}ignoring return value\char`\"{} when I call \hyperlink{gtest_8h_a853a3792807489591d3d4a2f2ff9359f}{R\+U\+N\+\_\+\+A\+L\+L\+\_\+\+T\+E\+S\+T\+S()}. Why?}

Some people had been ignoring the return value of {\ttfamily \hyperlink{gtest_8h_a853a3792807489591d3d4a2f2ff9359f}{R\+U\+N\+\_\+\+A\+L\+L\+\_\+\+T\+E\+S\+T\+S()}}. That is, instead of


\begin{DoxyCode}
1 return RUN\_ALL\_TESTS();
\end{DoxyCode}


they write


\begin{DoxyCode}
1 RUN\_ALL\_TESTS();
\end{DoxyCode}


This is wrong and dangerous. A test runner needs to see the return value of {\ttfamily \hyperlink{gtest_8h_a853a3792807489591d3d4a2f2ff9359f}{R\+U\+N\+\_\+\+A\+L\+L\+\_\+\+T\+E\+S\+T\+S()}} in order to determine if a test has passed. If your {\ttfamily \hyperlink{app_2main_8cpp_ae66f6b31b5ad750f1fe042a706a4e3d4}{main()}} function ignores it, your test will be considered successful even if it has a Google Test assertion failure. Very bad.

To help the users avoid this dangerous bug, the implementation of {\ttfamily \hyperlink{gtest_8h_a853a3792807489591d3d4a2f2ff9359f}{R\+U\+N\+\_\+\+A\+L\+L\+\_\+\+T\+E\+S\+T\+S()}} causes gcc to raise this warning, when the return value is ignored. If you see this warning, the fix is simple\+: just make sure its value is used as the return value of {\ttfamily \hyperlink{app_2main_8cpp_ae66f6b31b5ad750f1fe042a706a4e3d4}{main()}}.

\subsection*{My compiler complains that a constructor (or destructor) cannot return a value. What\textquotesingle{}s going on?}

Due to a peculiarity of C++, in order to support the syntax for streaming messages to an {\ttfamily A\+S\+S\+E\+R\+T\+\_\+$\ast$}, e.\+g.


\begin{DoxyCode}
1 ASSERT\_EQ(1, Foo()) << "blah blah" << foo;
\end{DoxyCode}


we had to give up using {\ttfamily A\+S\+S\+E\+R\+T$\ast$} and {\ttfamily F\+A\+I\+L$\ast$} (but not {\ttfamily E\+X\+P\+E\+C\+T$\ast$} and {\ttfamily A\+D\+D\+\_\+\+F\+A\+I\+L\+U\+R\+E$\ast$}) in constructors and destructors. The workaround is to move the content of your constructor/destructor to a private void member function, or switch to {\ttfamily E\+X\+P\+E\+C\+T\+\_\+$\ast$()} if that works. This section in the user\textquotesingle{}s guide explains it.

\subsection*{My set-\/up function is not called. Why?}

C++ is case-\/sensitive. It should be spelled as {\ttfamily Set\+Up()}. Did you spell it as {\ttfamily Setup()}?

Similarly, sometimes people spell {\ttfamily Set\+Up\+Test\+Case()} as {\ttfamily Setup\+Test\+Case()} and wonder why it\textquotesingle{}s never called.

\subsection*{How do I jump to the line of a failure in Emacs directly?}

Google Test\textquotesingle{}s failure message format is understood by Emacs and many other I\+D\+Es, like acme and X\+Code. If a Google Test message is in a compilation buffer in Emacs, then it\textquotesingle{}s clickable. You can now hit {\ttfamily enter} on a message to jump to the corresponding source code, or use {\ttfamily C-\/x}` to jump to the next failure.

\subsection*{I have several test cases which share the same test fixture logic, do I have to define a new test fixture class for each of them? This seems pretty tedious.}

You don\textquotesingle{}t have to. Instead of


\begin{DoxyCode}
1 class FooTest : public BaseTest \{\};
2 
3 TEST\_F(FooTest, Abc) \{ ... \}
4 TEST\_F(FooTest, Def) \{ ... \}
5 
6 class BarTest : public BaseTest \{\};
7 
8 TEST\_F(BarTest, Abc) \{ ... \}
9 TEST\_F(BarTest, Def) \{ ... \}
\end{DoxyCode}


you can simply {\ttfamily typedef} the test fixtures\+: 
\begin{DoxyCode}
1 typedef BaseTest FooTest;
2 
3 TEST\_F(FooTest, Abc) \{ ... \}
4 TEST\_F(FooTest, Def) \{ ... \}
5 
6 typedef BaseTest BarTest;
7 
8 TEST\_F(BarTest, Abc) \{ ... \}
9 TEST\_F(BarTest, Def) \{ ... \}
\end{DoxyCode}


\subsection*{The Google Test output is buried in a whole bunch of log messages. What do I do?}

The Google Test output is meant to be a concise and human-\/friendly report. If your test generates textual output itself, it will mix with the Google Test output, making it hard to read. However, there is an easy solution to this problem.

Since most log messages go to stderr, we decided to let Google Test output go to stdout. This way, you can easily separate the two using redirection. For example\+: 
\begin{DoxyCode}
1 ./my\_test > googletest\_output.txt
\end{DoxyCode}


\subsection*{Why should I prefer test fixtures over global variables?}

There are several good reasons\+:
\begin{DoxyEnumerate}
\item It\textquotesingle{}s likely your test needs to change the states of its global variables. This makes it difficult to keep side effects from escaping one test and contaminating others, making debugging difficult. By using fixtures, each test has a fresh set of variables that\textquotesingle{}s different (but with the same names). Thus, tests are kept independent of each other.
\end{DoxyEnumerate}
\begin{DoxyEnumerate}
\item Global variables pollute the global namespace.
\end{DoxyEnumerate}
\begin{DoxyEnumerate}
\item Test fixtures can be reused via subclassing, which cannot be done easily with global variables. This is useful if many test cases have something in common.
\end{DoxyEnumerate}

\subsection*{How do I test private class members without writing \hyperlink{gtest__prod_8h_a8d443b4cc1d87a7a17943b8fbdbf3910}{F\+R\+I\+E\+N\+D\+\_\+\+T\+E\+S\+T()}s?}

You should try to write testable code, which means classes should be easily tested from their public interface. One way to achieve this is the Pimpl idiom\+: you move all private members of a class into a helper class, and make all members of the helper class public.

You have several other options that don\textquotesingle{}t require using {\ttfamily F\+R\+I\+E\+N\+D\+\_\+\+T\+E\+ST}\+:
\begin{DoxyItemize}
\item Write the tests as members of the fixture class\+: 
\begin{DoxyCode}
1 class Foo \{
2   friend class FooTest;
3   ...
4 \};
5 
6 class FooTest : public ::testing::Test \{
7  protected:
8   ...
9   void Test1() \{...\} // This accesses private members of class Foo.
10   void Test2() \{...\} // So does this one.
11 \};
12 
13 TEST\_F(FooTest, Test1) \{
14   Test1();
15 \}
16 
17 TEST\_F(FooTest, Test2) \{
18   Test2();
19 \}
\end{DoxyCode}

\item In the fixture class, write accessors for the tested class\textquotesingle{} private members, then use the accessors in your tests\+: 
\begin{DoxyCode}
1 class Foo \{
2   friend class FooTest;
3   ...
4 \};
5 
6 class FooTest : public ::testing::Test \{
7  protected:
8   ...
9   T1 get\_private\_member1(Foo* obj) \{
10     return obj->private\_member1\_;
11   \}
12 \};
13 
14 TEST\_F(FooTest, Test1) \{
15   ...
16   get\_private\_member1(x)
17   ...
18 \}
\end{DoxyCode}

\item If the methods are declared {\bfseries protected}, you can change their access level in a test-\/only subclass\+: 
\begin{DoxyCode}
1 class YourClass \{
2   ...
3  protected: // protected access for testability.
4   int DoSomethingReturningInt();
5   ...
6 \};
7 
8 // in the your\_class\_test.cc file:
9 class TestableYourClass : public YourClass \{
10   ...
11  public: using YourClass::DoSomethingReturningInt; // changes access rights
12   ...
13 \};
14 
15 TEST\_F(YourClassTest, DoSomethingTest) \{
16   TestableYourClass obj;
17   assertEquals(expected\_value, obj.DoSomethingReturningInt());
18 \}
\end{DoxyCode}

\end{DoxyItemize}

\subsection*{How do I test private class static members without writing \hyperlink{gtest__prod_8h_a8d443b4cc1d87a7a17943b8fbdbf3910}{F\+R\+I\+E\+N\+D\+\_\+\+T\+E\+S\+T()}s?}

We find private static methods clutter the header file. They are implementation details and ideally should be kept out of a .h. So often I make them free functions instead.

Instead of\+: 
\begin{DoxyCode}
1 // foo.h
2 class Foo \{
3   ...
4  private:
5   static bool Func(int n);
6 \};
7 
8 // foo.cc
9 bool Foo::Func(int n) \{ ... \}
10 
11 // foo\_test.cc
12 EXPECT\_TRUE(Foo::Func(12345));
\end{DoxyCode}


You probably should better write\+: 
\begin{DoxyCode}
1 // foo.h
2 class Foo \{
3   ...
4 \};
5 
6 // foo.cc
7 namespace internal \{
8   bool Func(int n) \{ ... \}
9 \}
10 
11 // foo\_test.cc
12 namespace internal \{
13   bool Func(int n);
14 \}
15 
16 EXPECT\_TRUE(internal::Func(12345));
\end{DoxyCode}


\subsection*{I would like to run a test several times with different parameters. Do I need to write several similar copies of it?}

No. You can use a feature called \href{V1_7_AdvancedGuide.md#Value_Parameterized_Tests}{\tt value-\/parameterized tests} which lets you repeat your tests with different parameters, without defining it more than once.

\subsection*{How do I test a file that defines \hyperlink{app_2main_8cpp_ae66f6b31b5ad750f1fe042a706a4e3d4}{main()}?}

To test a {\ttfamily foo.\+cc} file, you need to compile and link it into your unit test program. However, when the file contains a definition for the {\ttfamily \hyperlink{app_2main_8cpp_ae66f6b31b5ad750f1fe042a706a4e3d4}{main()}} function, it will clash with the {\ttfamily \hyperlink{app_2main_8cpp_ae66f6b31b5ad750f1fe042a706a4e3d4}{main()}} of your unit test, and will result in a build error.

The right solution is to split it into three files\+:
\begin{DoxyEnumerate}
\item {\ttfamily foo.\+h} which contains the declarations,
\end{DoxyEnumerate}
\begin{DoxyEnumerate}
\item {\ttfamily foo.\+cc} which contains the definitions except {\ttfamily \hyperlink{app_2main_8cpp_ae66f6b31b5ad750f1fe042a706a4e3d4}{main()}}, and
\end{DoxyEnumerate}
\begin{DoxyEnumerate}
\item {\ttfamily foo\+\_\+main.\+cc} which contains nothing but the definition of {\ttfamily \hyperlink{app_2main_8cpp_ae66f6b31b5ad750f1fe042a706a4e3d4}{main()}}.
\end{DoxyEnumerate}

Then {\ttfamily foo.\+cc} can be easily tested.

If you are adding tests to an existing file and don\textquotesingle{}t want an intrusive change like this, there is a hack\+: just include the entire {\ttfamily foo.\+cc} file in your unit test. For example\+: 
\begin{DoxyCode}
1 // File foo\_unittest.cc
2 
3 // The headers section
4 ...
5 
6 // Renames main() in foo.cc to make room for the unit test main()
7 #define main FooMain
8 
9 #include "a/b/foo.cc"
10 
11 // The tests start here.
12 ...
\end{DoxyCode}


However, please remember this is a hack and should only be used as the last resort.

\subsection*{What can the statement argument in A\+S\+S\+E\+R\+T\+\_\+\+D\+E\+A\+T\+H() be?}

{\ttfamily A\+S\+S\+E\+R\+T\+\_\+\+D\+E\+A\+T\+H(\+\_\+statement\+\_\+, \+\_\+regex\+\_\+)} (or any death assertion macro) can be used wherever {\ttfamily \+\_\+statement\+\_\+} is valid. So basically {\ttfamily \+\_\+statement\+\_\+} can be any C++ statement that makes sense in the current context. In particular, it can reference global and/or local variables, and can be\+:
\begin{DoxyItemize}
\item a simple function call (often the case),
\item a complex expression, or
\item a compound statement.
\end{DoxyItemize}

\begin{quote}
Some examples are shown here\+: \end{quote}



\begin{DoxyCode}
1 // A death test can be a simple function call.
2 TEST(MyDeathTest, FunctionCall) \{
3   ASSERT\_DEATH(Xyz(5), "Xyz failed");
4 \}
5 
6 // Or a complex expression that references variables and functions.
7 TEST(MyDeathTest, ComplexExpression) \{
8   const bool c = Condition();
9   ASSERT\_DEATH((c ? Func1(0) : object2.Method("test")),
10                "(Func1|Method) failed");
11 \}
12 
13 // Death assertions can be used any where in a function. In
14 // particular, they can be inside a loop.
15 TEST(MyDeathTest, InsideLoop) \{
16   // Verifies that Foo(0), Foo(1), ..., and Foo(4) all die.
17   for (int i = 0; i < 5; i++) \{
18     EXPECT\_DEATH\_M(Foo(i), "Foo has \(\backslash\)\(\backslash\)d+ errors",
19                    ::testing::Message() << "where i is " << i);
20   \}
21 \}
22 
23 // A death assertion can contain a compound statement.
24 TEST(MyDeathTest, CompoundStatement) \{
25   // Verifies that at lease one of Bar(0), Bar(1), ..., and
26   // Bar(4) dies.
27   ASSERT\_DEATH(\{
28     for (int i = 0; i < 5; i++) \{
29       Bar(i);
30     \}
31   \},
32   "Bar has \(\backslash\)\(\backslash\)d+ errors");\}
\end{DoxyCode}


{\ttfamily googletest\+\_\+unittest.\+cc} contains more examples if you are interested.

\subsection*{What syntax does the regular expression in A\+S\+S\+E\+R\+T\+\_\+\+D\+E\+A\+TH use?}

On P\+O\+S\+IX systems, Google Test uses the P\+O\+S\+IX Extended regular expression syntax (\href{http://en.wikipedia.org/wiki/Regular_expression#POSIX_Extended_Regular_Expressions}{\tt http\+://en.\+wikipedia.\+org/wiki/\+Regular\+\_\+expression\#\+P\+O\+S\+I\+X\+\_\+\+Extended\+\_\+\+Regular\+\_\+\+Expressions}). On Windows, it uses a limited variant of regular expression syntax. For more details, see the \href{V1_7_AdvancedGuide.md#Regular_Expression_Syntax}{\tt regular expression syntax}.

\subsection*{I have a fixture class Foo, but \hyperlink{gtest_8h_a0ee66d464d1a06c20c1929cae09d8758}{T\+E\+S\+T\+\_\+\+F(\+Foo, Bar)} gives me error \char`\"{}no matching function for call to Foo\+::\+Foo()\char`\"{}. Why?}

Google Test needs to be able to create objects of your test fixture class, so it must have a default constructor. Normally the compiler will define one for you. However, there are cases where you have to define your own\+:
\begin{DoxyItemize}
\item If you explicitly declare a non-\/default constructor for class {\ttfamily Foo}, then you need to define a default constructor, even if it would be empty.
\item If {\ttfamily Foo} has a const non-\/static data member, then you have to define the default constructor {\itshape and} initialize the const member in the initializer list of the constructor. (Early versions of {\ttfamily gcc} doesn\textquotesingle{}t force you to initialize the const member. It\textquotesingle{}s a bug that has been fixed in {\ttfamily gcc 4}.)
\end{DoxyItemize}

\subsection*{Why does A\+S\+S\+E\+R\+T\+\_\+\+D\+E\+A\+TH complain about previous threads that were already joined?}

With the Linux pthread library, there is no turning back once you cross the line from single thread to multiple threads. The first time you create a thread, a manager thread is created in addition, so you get 3, not 2, threads. Later when the thread you create joins the main thread, the thread count decrements by 1, but the manager thread will never be killed, so you still have 2 threads, which means you cannot safely run a death test.

The new N\+P\+TL thread library doesn\textquotesingle{}t suffer from this problem, as it doesn\textquotesingle{}t create a manager thread. However, if you don\textquotesingle{}t control which machine your test runs on, you shouldn\textquotesingle{}t depend on this.

\subsection*{Why does Google Test require the entire test case, instead of individual tests, to be named F\+O\+O\+Death\+Test when it uses A\+S\+S\+E\+R\+T\+\_\+\+D\+E\+A\+TH?}

Google Test does not interleave tests from different test cases. That is, it runs all tests in one test case first, and then runs all tests in the next test case, and so on. Google Test does this because it needs to set up a test case before the first test in it is run, and tear it down afterwords. Splitting up the test case would require multiple set-\/up and tear-\/down processes, which is inefficient and makes the semantics unclean.

If we were to determine the order of tests based on test name instead of test case name, then we would have a problem with the following situation\+:


\begin{DoxyCode}
1 TEST\_F(FooTest, AbcDeathTest) \{ ... \}
2 TEST\_F(FooTest, Uvw) \{ ... \}
3 
4 TEST\_F(BarTest, DefDeathTest) \{ ... \}
5 TEST\_F(BarTest, Xyz) \{ ... \}
\end{DoxyCode}


Since {\ttfamily Foo\+Test.\+Abc\+Death\+Test} needs to run before {\ttfamily Bar\+Test.\+Xyz}, and we don\textquotesingle{}t interleave tests from different test cases, we need to run all tests in the {\ttfamily \hyperlink{classFooTest}{Foo\+Test}} case before running any test in the {\ttfamily Bar\+Test} case. This contradicts with the requirement to run {\ttfamily Bar\+Test.\+Def\+Death\+Test} before {\ttfamily Foo\+Test.\+Uvw}.

\subsection*{But I don\textquotesingle{}t like calling my entire test case F\+O\+O\+Death\+Test when it contains both death tests and non-\/death tests. What do I do?}

You don\textquotesingle{}t have to, but if you like, you may split up the test case into {\ttfamily \hyperlink{classFooTest}{Foo\+Test}} and {\ttfamily Foo\+Death\+Test}, where the names make it clear that they are related\+:


\begin{DoxyCode}
1 class FooTest : public ::testing::Test \{ ... \};
2 
3 TEST\_F(FooTest, Abc) \{ ... \}
4 TEST\_F(FooTest, Def) \{ ... \}
5 
6 typedef FooTest FooDeathTest;
7 
8 TEST\_F(FooDeathTest, Uvw) \{ ... EXPECT\_DEATH(...) ... \}
9 TEST\_F(FooDeathTest, Xyz) \{ ... ASSERT\_DEATH(...) ... \}
\end{DoxyCode}


\subsection*{The compiler complains about \char`\"{}no match for \textquotesingle{}operator$<$$<$\textquotesingle{}\char`\"{} when I use an assertion. What gives?}

If you use a user-\/defined type {\ttfamily Foo\+Type} in an assertion, you must make sure there is an {\ttfamily std\+::ostream\& operator$<$$<$(std\+::ostream\&, const Foo\+Type\&)} function defined such that we can print a value of {\ttfamily Foo\+Type}.

In addition, if {\ttfamily Foo\+Type} is declared in a name space, the {\ttfamily $<$$<$} operator also needs to be defined in the {\itshape same} name space.

\subsection*{How do I suppress the memory leak messages on Windows?}

Since the statically initialized Google Test singleton requires allocations on the heap, the Visual C++ memory leak detector will report memory leaks at the end of the program run. The easiest way to avoid this is to use the {\ttfamily \+\_\+\+Crt\+Mem\+Checkpoint} and {\ttfamily \+\_\+\+Crt\+Mem\+Dump\+All\+Objects\+Since} calls to not report any statically initialized heap objects. See M\+S\+DN for more details and additional heap check/debug routines.

\subsection*{I am building my project with Google Test in Visual Studio and all I\textquotesingle{}m getting is a bunch of linker errors (or warnings). Help!}

You may get a number of the following linker error or warnings if you attempt to link your test project with the Google Test library when your project and the are not built using the same compiler settings.


\begin{DoxyItemize}
\item L\+N\+K2005\+: symbol already defined in object
\item L\+N\+K4217\+: locally defined symbol \textquotesingle{}symbol\textquotesingle{} imported in function \textquotesingle{}function\textquotesingle{}
\item L\+N\+K4049\+: locally defined symbol \textquotesingle{}symbol\textquotesingle{} imported
\end{DoxyItemize}

The Google Test project (gtest.\+vcproj) has the Runtime Library option set to /\+MT (use multi-\/threaded static libraries, /\+M\+Td for debug). If your project uses something else, for example /\+MD (use multi-\/threaded D\+L\+Ls, /\+M\+Dd for debug), you need to change the setting in the Google Test project to match your project\textquotesingle{}s.

To update this setting open the project properties in the Visual Studio I\+DE then select the branch Configuration Properties $\vert$ C/\+C++ $\vert$ Code Generation and change the option \char`\"{}\+Runtime Library\char`\"{}. You may also try using gtest-\/md.\+vcproj instead of gtest.\+vcproj.

\subsection*{I put my tests in a library and Google Test doesn\textquotesingle{}t run them. What\textquotesingle{}s happening?}

Have you read a \href{V1_7_Primer.md#important-note-for-visual-c-users}{\tt warning} on the Google Test Primer page?

\subsection*{I want to use Google Test with Visual Studio but don\textquotesingle{}t know where to start.}

Many people are in your position and one of the posted his solution to our mailing list. Here is his link\+: \href{http://hassanjamilahmad.blogspot.com/2009/07/gtest-starters-help.html}{\tt http\+://hassanjamilahmad.\+blogspot.\+com/2009/07/gtest-\/starters-\/help.\+html}.

\subsection*{I am seeing compile errors mentioning std\+::type\+\_\+traits when I try to use Google Test on Solaris.}

Google Test uses parts of the standard C++ library that Sun\+Studio does not support. Our users reported success using alternative implementations. Try running the build after runing this commad\+:

`export CC=cc C\+XX=CC C\+X\+X\+F\+L\+A\+GS=\textquotesingle{}-\/library=stlport4\textquotesingle{}`

\subsection*{How can my code detect if it is running in a test?}

If you write code that sniffs whether it\textquotesingle{}s running in a test and does different things accordingly, you are leaking test-\/only logic into production code and there is no easy way to ensure that the test-\/only code paths aren\textquotesingle{}t run by mistake in production. Such cleverness also leads to \href{http://en.wikipedia.org/wiki/Unusual_software_bug#Heisenbug}{\tt Heisenbugs}. Therefore we strongly advise against the practice, and Google Test doesn\textquotesingle{}t provide a way to do it.

In general, the recommended way to cause the code to behave differently under test is \href{http://jamesshore.com/Blog/Dependency-Injection-Demystified.html}{\tt dependency injection}. You can inject different functionality from the test and from the production code. Since your production code doesn\textquotesingle{}t link in the for-\/test logic at all, there is no danger in accidentally running it.

However, if you {\itshape really}, {\itshape really}, {\itshape really} have no choice, and if you follow the rule of ending your test program names with {\ttfamily \+\_\+test}, you can use the {\itshape horrible} hack of sniffing your executable name ({\ttfamily argv\mbox{[}0\mbox{]}} in {\ttfamily \hyperlink{app_2main_8cpp_ae66f6b31b5ad750f1fe042a706a4e3d4}{main()}}) to know whether the code is under test.

\subsection*{Google Test defines a macro that clashes with one defined by another library. How do I deal with that?}

In C++, macros don\textquotesingle{}t obey namespaces. Therefore two libraries that both define a macro of the same name will clash if you {\ttfamily \#include} both definitions. In case a Google Test macro clashes with another library, you can force Google Test to rename its macro to avoid the conflict.

Specifically, if both Google Test and some other code define macro {\ttfamily F\+OO}, you can add 
\begin{DoxyCode}
1 -DGTEST\_DONT\_DEFINE\_FOO=1
\end{DoxyCode}
 to the compiler flags to tell Google Test to change the macro\textquotesingle{}s name from {\ttfamily F\+OO} to {\ttfamily G\+T\+E\+S\+T\+\_\+\+F\+OO}. For example, with {\ttfamily -\/\+D\+G\+T\+E\+S\+T\+\_\+\+D\+O\+N\+T\+\_\+\+D\+E\+F\+I\+N\+E\+\_\+\+T\+E\+ST=1}, you\textquotesingle{}ll need to write 
\begin{DoxyCode}
1 GTEST\_TEST(SomeTest, DoesThis) \{ ... \}
\end{DoxyCode}
 instead of 
\begin{DoxyCode}
1 TEST(SomeTest, DoesThis) \{ ... \}
\end{DoxyCode}
 in order to define a test.

Currently, the following {\ttfamily T\+E\+ST}, {\ttfamily F\+A\+IL}, {\ttfamily S\+U\+C\+C\+E\+ED}, and the basic comparison assertion macros can have alternative names. You can see the full list of covered macros \href{http://www.google.com/codesearch?q=if+!GTEST_DONT_DEFINE_\w%2B+package:http://googletest\.googlecode\.com+file:/include/gtest/gtest.h}{\tt here}. More information can be found in the \char`\"{}\+Avoiding Macro Name Clashes\char`\"{} section of the R\+E\+A\+D\+ME file.

\subsection*{Is it OK if I have two separate {\ttfamily \hyperlink{gtest_8h_ad8b332753515c0ab8baada563c2547eb}{T\+E\+S\+T(\+Foo, Bar)}} test methods defined in different namespaces?}

Yes.

The rule is {\bfseries all test methods in the same test case must use the same fixture class}. This means that the following is {\bfseries allowed} because both tests use the same fixture class ({\ttfamily \hyperlink{classtesting_1_1Test}{testing\+::\+Test}}).


\begin{DoxyCode}
1 namespace foo \{
2 TEST(CoolTest, DoSomething) \{
3   SUCCEED();
4 \}
5 \}  // namespace foo
6 
7 namespace bar \{
8 TEST(CoolTest, DoSomething) \{
9   SUCCEED();
10 \}
11 \}  // namespace foo
\end{DoxyCode}


However, the following code is {\bfseries not allowed} and will produce a runtime error from Google Test because the test methods are using different test fixture classes with the same test case name.


\begin{DoxyCode}
1 namespace foo \{
2 class CoolTest : public ::testing::Test \{\};  // Fixture foo::CoolTest
3 TEST\_F(CoolTest, DoSomething) \{
4   SUCCEED();
5 \}
6 \}  // namespace foo
7 
8 namespace bar \{
9 class CoolTest : public ::testing::Test \{\};  // Fixture: bar::CoolTest
10 TEST\_F(CoolTest, DoSomething) \{
11   SUCCEED();
12 \}
13 \}  // namespace foo
\end{DoxyCode}


\subsection*{How do I build Google Testing Framework with Xcode 4?}

If you try to build Google Test\textquotesingle{}s Xcode project with Xcode 4.\+0 or later, you may encounter an error message that looks like \char`\"{}\+Missing S\+D\+K in target gtest\textbackslash{}\+\_\+framework\+: /\+Developer/\+S\+D\+Ks/\+Mac\+O\+S\+X10.\+4u.\+sdk\char`\"{}. That means that Xcode does not support the S\+DK the project is targeting. See the Xcode section in the ../../\+R\+E\+A\+D\+ME.MD \char`\"{}\+R\+E\+A\+D\+M\+E\char`\"{} file on how to resolve this.

\subsection*{My question is not covered in your F\+A\+Q!}

If you cannot find the answer to your question in this F\+AQ, there are some other resources you can use\+:


\begin{DoxyEnumerate}
\item read other \href{http://code.google.com/p/googletest/w/list}{\tt wiki pages},
\end{DoxyEnumerate}
\begin{DoxyEnumerate}
\item search the mailing list \href{http://groups.google.com/group/googletestframework/topics}{\tt archive},
\end{DoxyEnumerate}
\begin{DoxyEnumerate}
\item ask it on \href{mailto:googletestframework@googlegroups.com}{\tt googletestframework@googlegroups.\+com} and someone will answer it (to prevent spam, we require you to join the \href{http://groups.google.com/group/googletestframework}{\tt discussion group} before you can post.).
\end{DoxyEnumerate}

Please note that creating an issue in the \href{http://code.google.com/p/googletest/issues/list}{\tt issue tracker} is {\itshape not} a good way to get your answer, as it is monitored infrequently by a very small number of people.

When asking a question, it\textquotesingle{}s helpful to provide as much of the following information as possible (people cannot help you if there\textquotesingle{}s not enough information in your question)\+:


\begin{DoxyItemize}
\item the version (or the revision number if you check out from S\+VN directly) of Google Test you use (Google Test is under active development, so it\textquotesingle{}s possible that your problem has been solved in a later version),
\item your operating system,
\item the name and version of your compiler,
\item the complete command line flags you give to your compiler,
\item the complete compiler error messages (if the question is about compilation),
\item the {\itshape actual} code (ideally, a minimal but complete program) that has the problem you encounter. 
\end{DoxyItemize}
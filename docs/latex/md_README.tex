A$\ast$ Path Planning Component Design for Acme.

\href{https://travis-ci.org/SonamYeshe/AStarPathPlanning}{\tt } \subsection*{\href{https://coveralls.io/github/SonamYeshe/AStarPathPlanning?branch=master}{\tt } }

\subsection*{Overview and purpose of the project}

While Acme robot is on automation mode and plan to move from an origin to a goal, this project helps to find the path with the shortest path.

At each iteration of its main loop, A$\ast$ needs to determine which of its partial paths to expand into one or more longer paths. It does so based on an estimate of the cost (total weight) still to go to the goal node. Specifically, A$\ast$ selects the path that minimizes f(n)=g(n)+h(n) where n is the last node on the path, g(n) is the cost of the path from the start node to n, and h(n) is a heuristic that estimates the cost of the cheapest path from n to the goal. The heuristic is problem-\/specific. In this project, we choose the direct distance from the current position to the goal.

\subsection*{Performance example}

Illustration of A$\ast$ search for finding path from a start node to a goal node in a robot motion planning problem. The empty circles represent the nodes in the open set, i.\+e., those that remain to be explored, and the filled ones are in the closed set. Color on each closed node indicates the distance from the start\+: the greener, the farther. One can first see the A$\ast$ moving in a straight line in the direction of the goal, then when hitting the obstacle, it explores alternative routes through the nodes from the open set. Gif can be seen from the following link. 

 

\subsection*{License}

G\+NU General Public License v3.\+0

Permissions of this strong copyleft license are conditioned on making available complete source code of licensed works and modifications, which include larger works using a licensed work, under the same license. Copyright and license notices must be preserved. Contributors provide an express grant of patent rights.

\subsection*{Dependencies}


\begin{DoxyEnumerate}
\item eclipse C\+DT I\+DE (4.\+7).
\end{DoxyEnumerate}

\subsection*{Backlog google spreadsheet}

\href{https://docs.google.com/a/terpmail.umd.edu/spreadsheets/d/1j3ytA1mPyvJpSjNRXjjnHxmMqRgYix9c8MPspWwJh78/edit?usp=sharing}{\tt https\+://docs.\+google.\+com/a/terpmail.\+umd.\+edu/spreadsheets/d/1j3yt\+A1m\+Pyv\+Jp\+Sj\+N\+R\+Xjjn\+Hxm\+Mq\+Rg\+Yix9c8\+M\+Psp\+Ww\+Jh78/edit?usp=sharing}

\subsection*{Working with Eclipse I\+DE}

\subsection*{Installation}

In your Eclipse workspace directory (or create a new one), checkout the repo (and submodules) 
\begin{DoxyCode}
1 mkdir -p ~/workspace
2 cd ~/workspace
3 git clone --recursive https://github.com/SonamYeshe/AStarPathPlanning.git
\end{DoxyCode}


In your work directory, use cmake to create an Eclipse project for an \mbox{[}out-\/of-\/source build\mbox{]} of A\+Star\+Path\+Planning


\begin{DoxyCode}
1 cd ~/workspace
2 mkdir -p AStarPathPlanning-eclipse
3 cd AStarPathPlanning-eclipse
4 cmake -G "Eclipse CDT4 - Unix Makefiles" -D CMAKE\_BUILD\_TYPE=Debug -D CMAKE\_ECLIPSE\_VERSION=4.7.0 -D
       CMAKE\_CXX\_COMPILER\_ARG1=-std=c++14 ../AStarPathPlanning/
\end{DoxyCode}


\subsection*{Import}

Open Eclipse, go to File -\/$>$ Import -\/$>$ General -\/$>$ Existing Projects into Workspace -\/$>$ Select \char`\"{}\+A\+Star\+Path\+Planning-\/eclipse\char`\"{} directory created previously as root directory -\/$>$ Finish

\section*{Edit}

Source files may be edited under the \char`\"{}\mbox{[}\+Source Directory\mbox{]}\char`\"{} label in the Project Explorer.

\subsection*{Build}

To build the project, in Eclipse, unfold A\+Star\+Path\+Planning-\/eclipse project in Project Explorer, unfold Build Targets, double click on \char`\"{}all\char`\"{} to build all projects.

\subsection*{Run}


\begin{DoxyEnumerate}
\item In Eclipse, right click on the A\+Star\+Path\+Planning-\/eclipse in Project Explorer, select Run As -\/$>$ Local C/\+C++ Application
\item Choose the binaries to run (e.\+g. shell-\/app, cpp-\/test for unit testing) A\+Star\+Path\+Planning-\/eclipse in Project Explorer, select Debug As -\/$>$ Local C/\+C++ Application, choose the binaries to run (e.\+g. shell-\/app).
\item If prompt to \char`\"{}\+Confirm Perspective Switch\char`\"{}, select yes.
\item Program will break at the breakpoint you set.
\item Press Step Into (F5), Step Over (F6), Step Return (F7) to step/debug your program.
\item Right click on the variable in editor to add watch expression to watch the variable in debugger window.
\item Press Terminate icon to terminate debugging and press C/\+C++ icon to switch back to C/\+C++ perspetive view (or Windows-\/$>$Perspective-\/$>$Open Perspective-\/$>$C/\+C++).
\end{DoxyEnumerate}

\subsection*{Plugins}


\begin{DoxyItemize}
\item Cpp\+Ch\+Eclipse

To install and run cppcheck in Eclipse
\begin{DoxyEnumerate}
\item In Eclipse, go to Window -\/$>$ Preferences -\/$>$ C/\+C++ -\/$>$ cppcheclipse. Set cppcheck binary path to \char`\"{}/usr/bin/cppcheck\char`\"{}.
\item To run C\+P\+P\+Check on a project, right click on the project name in the Project Explorer and choose cppcheck -\/$>$ Run cppcheck.
\end{DoxyEnumerate}
\item cpplint

To install and run from terminal
\begin{DoxyEnumerate}
\item wget \href{https://raw.githubusercontent.com/google/styleguide/gh-pages/cpplint/cpplint.py}{\tt https\+://raw.\+githubusercontent.\+com/google/styleguide/gh-\/pages/cpplint/cpplint.\+py}
\item Change permission to executable\+: chmod +x cpplint.\+py
\item cd $<$repository$>$
\item Run cpplint
\end{DoxyEnumerate}

Example run with A\+Star\+Path\+Planning-\/eclipse (cpplint.\+py is in the parent directory)\+:

cd cpp-\/boilerplate

../cpplint.py --extensions=h,hpp,cpp \$( find . -\/name $\ast$.h -\/or -\/name $\ast$.hpp -\/or -\/name $\ast$.cpp $\vert$ grep -\/vE -\/e \char`\"{}$^\wedge$./build/\char`\"{} -\/e \char`\"{}$^\wedge$./vendor/\char`\"{} )
\item doxygen

Simply run \$ doxygen my\+\_\+proj.\+conf in a new terminal under repo to update the doxygen documentation.
\item Google C++ Sytle

To include and use Google C++ Style formatter in Eclipse
\begin{DoxyEnumerate}
\item In Eclipse, go to Window -\/$>$ Preferences -\/$>$ C/\+C++ -\/$>$ Code Style -\/$>$ Formatter. Import \href{https://raw.githubusercontent.com/google/styleguide/gh-pages/eclipse-cpp-google-style.xml}{\tt eclipse-\/cpp-\/google-\/style} and apply.
\item To use Google C++ style formatter, right click on the source code or folder in Project Explorer and choose Source -\/$>$ Format
\end{DoxyEnumerate}
\item Git

It is possible to manage version control through Eclipse and the git plugin, but it typically requires creating another project. If you\textquotesingle{}re interested in this, try it out yourself and contact me on Canvas. 
\end{DoxyItemize}
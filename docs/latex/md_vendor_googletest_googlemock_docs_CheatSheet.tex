\subsection*{Mocking a Normal Class}

Given 
\begin{DoxyCode}
1 class Foo \{
2   ...
3   virtual ~Foo();
4   virtual int GetSize() const = 0;
5   virtual string Describe(const char* name) = 0;
6   virtual string Describe(int type) = 0;
7   virtual bool Process(Bar elem, int count) = 0;
8 \};
\end{DoxyCode}
 (note that {\ttfamily $\sim$\+Foo()} {\bfseries must} be virtual) we can define its mock as 
\begin{DoxyCode}
1 #include "gmock/gmock.h"
2 
3 class MockFoo : public Foo \{
4   MOCK\_CONST\_METHOD0(GetSize, int());
5   MOCK\_METHOD1(Describe, string(const char* name));
6   MOCK\_METHOD1(Describe, string(int type));
7   MOCK\_METHOD2(Process, bool(Bar elem, int count));
8 \};
\end{DoxyCode}


To create a \char`\"{}nice\char`\"{} mock object which ignores all uninteresting calls, or a \char`\"{}strict\char`\"{} mock object, which treats them as failures\+: 
\begin{DoxyCode}
1 NiceMock<MockFoo> nice\_foo;     // The type is a subclass of MockFoo.
2 StrictMock<MockFoo> strict\_foo; // The type is a subclass of MockFoo.
\end{DoxyCode}


\subsection*{Mocking a Class Template}

To mock 
\begin{DoxyCode}
1 template <typename Elem>
2 class StackInterface \{
3  public:
4   ...
5   virtual ~StackInterface();
6   virtual int GetSize() const = 0;
7   virtual void Push(const Elem& x) = 0;
8 \};
\end{DoxyCode}
 (note that {\ttfamily $\sim$\+Stack\+Interface()} {\bfseries must} be virtual) just append {\ttfamily \+\_\+T} to the {\ttfamily M\+O\+C\+K\+\_\+$\ast$} macros\+: 
\begin{DoxyCode}
1 template <typename Elem>
2 class MockStack : public StackInterface<Elem> \{
3  public:
4   ...
5   MOCK\_CONST\_METHOD0\_T(GetSize, int());
6   MOCK\_METHOD1\_T(Push, void(const Elem& x));
7 \};
\end{DoxyCode}


\subsection*{Specifying Calling Conventions for \hyperlink{classMock}{Mock} Functions}

If your mock function doesn\textquotesingle{}t use the default calling convention, you can specify it by appending {\ttfamily \+\_\+\+W\+I\+T\+H\+\_\+\+C\+A\+L\+L\+T\+Y\+PE} to any of the macros described in the previous two sections and supplying the calling convention as the first argument to the macro. For example, 
\begin{DoxyCode}
1 MOCK\_METHOD\_1\_WITH\_CALLTYPE(STDMETHODCALLTYPE, Foo, bool(int n));
2 MOCK\_CONST\_METHOD2\_WITH\_CALLTYPE(STDMETHODCALLTYPE, Bar, int(double x, double y));
\end{DoxyCode}
 where {\ttfamily S\+T\+D\+M\+E\+T\+H\+O\+D\+C\+A\+L\+L\+T\+Y\+PE} is defined by {\ttfamily $<$objbase.\+h$>$} on Windows.

\section*{Using Mocks in Tests}

The typical flow is\+:
\begin{DoxyEnumerate}
\item Import the Google \hyperlink{classMock}{Mock} names you need to use. All Google \hyperlink{classMock}{Mock} names are in the {\ttfamily testing} namespace unless they are macros or otherwise noted.
\end{DoxyEnumerate}
\begin{DoxyEnumerate}
\item Create the mock objects.
\end{DoxyEnumerate}
\begin{DoxyEnumerate}
\item Optionally, set the default actions of the mock objects.
\end{DoxyEnumerate}
\begin{DoxyEnumerate}
\item Set your expectations on the mock objects (How will they be called? What wil they do?).
\end{DoxyEnumerate}
\begin{DoxyEnumerate}
\item Exercise code that uses the mock objects; if necessary, check the result using \href{../../googletest/}{\tt Google Test} assertions.
\end{DoxyEnumerate}
\begin{DoxyEnumerate}
\item When a mock objects is destructed, Google \hyperlink{classMock}{Mock} automatically verifies that all expectations on it have been satisfied.
\end{DoxyEnumerate}

Here is an example\+: 
\begin{DoxyCode}
1 using ::testing::Return;                            // #1
2 
3 TEST(BarTest, DoesThis) \{
4   MockFoo foo;                                    // #2
5 
6   ON\_CALL(foo, GetSize())                         // #3
7       .WillByDefault(Return(1));
8   // ... other default actions ...
9 
10   EXPECT\_CALL(foo, Describe(5))                   // #4
11       .Times(3)
12       .WillRepeatedly(Return("Category 5"));
13   // ... other expectations ...
14 
15   EXPECT\_EQ("good", MyProductionFunction(&foo));  // #5
16 \}                                                 // #6
\end{DoxyCode}


\section*{Setting Default Actions}

Google \hyperlink{classMock}{Mock} has a {\bfseries built-\/in default action} for any function that returns {\ttfamily void}, {\ttfamily bool}, a numeric value, or a pointer.

To customize the default action for functions with return type {\ttfamily T} globally\+: 
\begin{DoxyCode}
1 using ::testing::DefaultValue;
2 
3 // Sets the default value to be returned. T must be CopyConstructible.
4 DefaultValue<T>::Set(value);
5 // Sets a factory. Will be invoked on demand. T must be MoveConstructible.
6 //   T MakeT();
7 DefaultValue<T>::SetFactory(&MakeT);
8 // ... use the mocks ...
9 // Resets the default value.
10 DefaultValue<T>::Clear();
\end{DoxyCode}


To customize the default action for a particular method, use {\ttfamily \hyperlink{gmock-spec-builders_8h_a5b12ae6cf84f0a544ca811b380c37334}{O\+N\+\_\+\+C\+A\+L\+L()}}\+: 
\begin{DoxyCode}
1 ON\_CALL(mock\_object, method(matchers))
2     .With(multi\_argument\_matcher)  ?
3     .WillByDefault(action);
\end{DoxyCode}


\section*{Setting Expectations}

{\ttfamily \hyperlink{gmock-spec-builders_8h_a535a6156de72c1a2e25a127e38ee5232}{E\+X\+P\+E\+C\+T\+\_\+\+C\+A\+L\+L()}} sets {\bfseries expectations} on a mock method (How will it be called? What will it do?)\+: 
\begin{DoxyCode}
1 EXPECT\_CALL(mock\_object, method(matchers))
2     .With(multi\_argument\_matcher)  ?
3     .Times(cardinality)            ?
4     .InSequence(sequences)         *
5     .After(expectations)           *
6     .WillOnce(action)              *
7     .WillRepeatedly(action)        ?
8     .RetiresOnSaturation();        ?
\end{DoxyCode}


If {\ttfamily Times()} is omitted, the cardinality is assumed to be\+:


\begin{DoxyItemize}
\item {\ttfamily Times(1)} when there is neither {\ttfamily Will\+Once()} nor {\ttfamily Will\+Repeatedly()};
\item {\ttfamily Times(n)} when there are {\ttfamily n Will\+Once()}s but no {\ttfamily Will\+Repeatedly()}, where {\ttfamily n} $>$= 1; or
\item {\ttfamily Times(\+At\+Least(n))} when there are {\ttfamily n Will\+Once()}s and a {\ttfamily Will\+Repeatedly()}, where {\ttfamily n} $>$= 0.
\end{DoxyItemize}

A method with no {\ttfamily \hyperlink{gmock-spec-builders_8h_a535a6156de72c1a2e25a127e38ee5232}{E\+X\+P\+E\+C\+T\+\_\+\+C\+A\+L\+L()}} is free to be invoked {\itshape any number of times}, and the default action will be taken each time.

\section*{Matchers}

A {\bfseries matcher} matches a {\itshape single} argument. You can use it inside {\ttfamily \hyperlink{gmock-spec-builders_8h_a5b12ae6cf84f0a544ca811b380c37334}{O\+N\+\_\+\+C\+A\+L\+L()}} or {\ttfamily \hyperlink{gmock-spec-builders_8h_a535a6156de72c1a2e25a127e38ee5232}{E\+X\+P\+E\+C\+T\+\_\+\+C\+A\+L\+L()}}, or use it to validate a value directly\+:

\tabulinesep=1mm
\begin{longtabu} spread 0pt [c]{*2{|X[-1]}|}
\hline
\rowcolor{\tableheadbgcolor}{\bf {\ttfamily \hyperlink{gmock-matchers_8h_ac31e206123aa702e1152bb2735b31409}{E\+X\+P\+E\+C\+T\+\_\+\+T\+H\+A\+T(value, matcher)}} }&{\bf Asserts that {\ttfamily value} matches {\ttfamily matcher}.  }\\\cline{1-2}
\endfirsthead
\hline
\endfoot
\hline
\rowcolor{\tableheadbgcolor}{\bf {\ttfamily \hyperlink{gmock-matchers_8h_ac31e206123aa702e1152bb2735b31409}{E\+X\+P\+E\+C\+T\+\_\+\+T\+H\+A\+T(value, matcher)}} }&{\bf Asserts that {\ttfamily value} matches {\ttfamily matcher}.  }\\\cline{1-2}
\endhead
{\ttfamily \hyperlink{gmock-matchers_8h_a41d888579850c16583baea33ee8d057e}{A\+S\+S\+E\+R\+T\+\_\+\+T\+H\+A\+T(value, matcher)}} &The same as {\ttfamily \hyperlink{gmock-matchers_8h_ac31e206123aa702e1152bb2735b31409}{E\+X\+P\+E\+C\+T\+\_\+\+T\+H\+A\+T(value, matcher)}}, except that it generates a {\bfseries fatal} failure. \\\cline{1-2}
\end{longtabu}
Built-\/in matchers (where {\ttfamily argument} is the function argument) are divided into several categories\+:

\subsection*{Wildcard}

\tabulinesep=1mm
\begin{longtabu} spread 0pt [c]{*2{|X[-1]}|}
\hline
\rowcolor{\tableheadbgcolor}{\bf {\ttfamily \+\_\+}}&{\bf {\ttfamily argument} can be any value of the correct type.  }\\\cline{1-2}
\endfirsthead
\hline
\endfoot
\hline
\rowcolor{\tableheadbgcolor}{\bf {\ttfamily \+\_\+}}&{\bf {\ttfamily argument} can be any value of the correct type.  }\\\cline{1-2}
\endhead
{\ttfamily A$<$type$>$()} or {\ttfamily An$<$type$>$()}&{\ttfamily argument} can be any value of type {\ttfamily type}. \\\cline{1-2}
\end{longtabu}
\subsection*{Generic Comparison}

\tabulinesep=1mm
\begin{longtabu} spread 0pt [c]{*2{|X[-1]}|}
\hline
\rowcolor{\tableheadbgcolor}{\bf {\ttfamily Eq(value)} or {\ttfamily value}}&{\bf {\ttfamily argument == value}  }\\\cline{1-2}
\endfirsthead
\hline
\endfoot
\hline
\rowcolor{\tableheadbgcolor}{\bf {\ttfamily Eq(value)} or {\ttfamily value}}&{\bf {\ttfamily argument == value}  }\\\cline{1-2}
\endhead
{\ttfamily Ge(value)} &{\ttfamily argument $>$= value} \\\cline{1-2}
{\ttfamily Gt(value)} &{\ttfamily argument $>$ value} \\\cline{1-2}
{\ttfamily Le(value)} &{\ttfamily argument $<$= value} \\\cline{1-2}
{\ttfamily Lt(value)} &{\ttfamily argument $<$ value} \\\cline{1-2}
{\ttfamily Ne(value)} &{\ttfamily argument != value} \\\cline{1-2}
{\ttfamily \hyperlink{namespacetesting_a56ffb1a169c14ce585fc5bed32add2db}{Is\+Null()}} &{\ttfamily argument} is a {\ttfamily N\+U\+LL} pointer (raw or smart). \\\cline{1-2}
{\ttfamily \hyperlink{namespacetesting_a39d1f92b53b8b2a0b6db6a22ac146416}{Not\+Null()}} &{\ttfamily argument} is a non-\/null pointer (raw or smart). \\\cline{1-2}
{\ttfamily Ref(variable)} &{\ttfamily argument} is a reference to {\ttfamily variable}. \\\cline{1-2}
{\ttfamily Typed\+Eq$<$type$>$(value)}&{\ttfamily argument} has type {\ttfamily type} and is equal to {\ttfamily value}. You may need to use this instead of {\ttfamily Eq(value)} when the mock function is overloaded. \\\cline{1-2}
\end{longtabu}
Except {\ttfamily \hyperlink{namespacetesting_a0a4a364121ea3fa656a112f1c2e6b7a4}{Ref()}}, these matchers make a {\itshape copy} of {\ttfamily value} in case it\textquotesingle{}s modified or destructed later. If the compiler complains that {\ttfamily value} doesn\textquotesingle{}t have a public copy constructor, try wrap it in {\ttfamily \hyperlink{namespacetesting_aaee6d42dcd69de6e7a1459c5c71222c3}{By\+Ref()}}, e.\+g. {\ttfamily Eq(\+By\+Ref(non\+\_\+copyable\+\_\+value))}. If you do that, make sure {\ttfamily non\+\_\+copyable\+\_\+value} is not changed afterwards, or the meaning of your matcher will be changed.

\subsection*{Floating-\/\+Point Matchers}

\tabulinesep=1mm
\begin{longtabu} spread 0pt [c]{*2{|X[-1]}|}
\hline
\rowcolor{\tableheadbgcolor}{\bf {\ttfamily Double\+Eq(a\+\_\+double)}}&{\bf {\ttfamily argument} is a {\ttfamily double} value approximately equal to {\ttfamily a\+\_\+double}, treating two Na\+Ns as unequal.  }\\\cline{1-2}
\endfirsthead
\hline
\endfoot
\hline
\rowcolor{\tableheadbgcolor}{\bf {\ttfamily Double\+Eq(a\+\_\+double)}}&{\bf {\ttfamily argument} is a {\ttfamily double} value approximately equal to {\ttfamily a\+\_\+double}, treating two Na\+Ns as unequal.  }\\\cline{1-2}
\endhead
{\ttfamily Float\+Eq(a\+\_\+float)} &{\ttfamily argument} is a {\ttfamily float} value approximately equal to {\ttfamily a\+\_\+float}, treating two Na\+Ns as unequal. \\\cline{1-2}
{\ttfamily Nan\+Sensitive\+Double\+Eq(a\+\_\+double)}&{\ttfamily argument} is a {\ttfamily double} value approximately equal to {\ttfamily a\+\_\+double}, treating two Na\+Ns as equal. \\\cline{1-2}
{\ttfamily Nan\+Sensitive\+Float\+Eq(a\+\_\+float)}&{\ttfamily argument} is a {\ttfamily float} value approximately equal to {\ttfamily a\+\_\+float}, treating two Na\+Ns as equal. \\\cline{1-2}
\end{longtabu}
The above matchers use U\+L\+P-\/based comparison (the same as used in \href{../../googletest/}{\tt Google Test}). They automatically pick a reasonable error bound based on the absolute value of the expected value. {\ttfamily \hyperlink{namespacetesting_a1f49f9d97f03bfa4db26888f03486a9f}{Double\+Eq()}} and {\ttfamily \hyperlink{namespacetesting_ac75507edb7998957b48fb17a9b8a020b}{Float\+Eq()}} conform to the I\+E\+EE standard, which requires comparing two Na\+Ns for equality to return false. The {\ttfamily Nan\+Sensitive$\ast$} version instead treats two Na\+Ns as equal, which is often what a user wants.

\tabulinesep=1mm
\begin{longtabu} spread 0pt [c]{*2{|X[-1]}|}
\hline
\rowcolor{\tableheadbgcolor}{\bf {\ttfamily Double\+Near(a\+\_\+double, max\+\_\+abs\+\_\+error)}}&{\bf {\ttfamily argument} is a {\ttfamily double} value close to {\ttfamily a\+\_\+double} (absolute error $<$= {\ttfamily max\+\_\+abs\+\_\+error}), treating two Na\+Ns as unequal.  }\\\cline{1-2}
\endfirsthead
\hline
\endfoot
\hline
\rowcolor{\tableheadbgcolor}{\bf {\ttfamily Double\+Near(a\+\_\+double, max\+\_\+abs\+\_\+error)}}&{\bf {\ttfamily argument} is a {\ttfamily double} value close to {\ttfamily a\+\_\+double} (absolute error $<$= {\ttfamily max\+\_\+abs\+\_\+error}), treating two Na\+Ns as unequal.  }\\\cline{1-2}
\endhead
{\ttfamily Float\+Near(a\+\_\+float, max\+\_\+abs\+\_\+error)} &{\ttfamily argument} is a {\ttfamily float} value close to {\ttfamily a\+\_\+float} (absolute error $<$= {\ttfamily max\+\_\+abs\+\_\+error}), treating two Na\+Ns as unequal. \\\cline{1-2}
{\ttfamily Nan\+Sensitive\+Double\+Near(a\+\_\+double, max\+\_\+abs\+\_\+error)}&{\ttfamily argument} is a {\ttfamily double} value close to {\ttfamily a\+\_\+double} (absolute error $<$= {\ttfamily max\+\_\+abs\+\_\+error}), treating two Na\+Ns as equal. \\\cline{1-2}
{\ttfamily Nan\+Sensitive\+Float\+Near(a\+\_\+float, max\+\_\+abs\+\_\+error)}&{\ttfamily argument} is a {\ttfamily float} value close to {\ttfamily a\+\_\+float} (absolute error $<$= {\ttfamily max\+\_\+abs\+\_\+error}), treating two Na\+Ns as equal. \\\cline{1-2}
\end{longtabu}
\subsection*{String Matchers}

The {\ttfamily argument} can be either a C string or a C++ string object\+:

\tabulinesep=1mm
\begin{longtabu} spread 0pt [c]{*2{|X[-1]}|}
\hline
\rowcolor{\tableheadbgcolor}{\bf {\ttfamily Contains\+Regex(string)}}&{\bf {\ttfamily argument} matches the given regular expression.  }\\\cline{1-2}
\endfirsthead
\hline
\endfoot
\hline
\rowcolor{\tableheadbgcolor}{\bf {\ttfamily Contains\+Regex(string)}}&{\bf {\ttfamily argument} matches the given regular expression.  }\\\cline{1-2}
\endhead
{\ttfamily Ends\+With(suffix)} &{\ttfamily argument} ends with string {\ttfamily suffix}. \\\cline{1-2}
{\ttfamily Has\+Substr(string)} &{\ttfamily argument} contains {\ttfamily string} as a sub-\/string. \\\cline{1-2}
{\ttfamily Matches\+Regex(string)} &{\ttfamily argument} matches the given regular expression with the match starting at the first character and ending at the last character. \\\cline{1-2}
{\ttfamily Starts\+With(prefix)} &{\ttfamily argument} starts with string {\ttfamily prefix}. \\\cline{1-2}
{\ttfamily Str\+Case\+Eq(string)} &{\ttfamily argument} is equal to {\ttfamily string}, ignoring case. \\\cline{1-2}
{\ttfamily Str\+Case\+Ne(string)} &{\ttfamily argument} is not equal to {\ttfamily string}, ignoring case. \\\cline{1-2}
{\ttfamily Str\+Eq(string)} &{\ttfamily argument} is equal to {\ttfamily string}. \\\cline{1-2}
{\ttfamily Str\+Ne(string)} &{\ttfamily argument} is not equal to {\ttfamily string}. \\\cline{1-2}
\end{longtabu}
{\ttfamily \hyperlink{namespacetesting_a899838630a71376aa071dfd7c500f2ca}{Contains\+Regex()}} and {\ttfamily \hyperlink{namespacetesting_a4dac232f315edc259b62ce88e413b107}{Matches\+Regex()}} use the regular expression syntax defined \href{../../googletest/docs/AdvancedGuide.md#regular-expression-syntax}{\tt here}. {\ttfamily \hyperlink{namespacetesting_addbdb8c0902ab09cade24fa6db7c0073}{Str\+Case\+Eq()}}, {\ttfamily \hyperlink{namespacetesting_a9596fd67bdd5df195bc54382f91a1906}{Str\+Case\+Ne()}}, {\ttfamily \hyperlink{namespacetesting_a154de81376ae7d584c6e5a619b45fb09}{Str\+Eq()}}, and {\ttfamily \hyperlink{namespacetesting_a6f6b8af3d19cb969df0eb995c47a14d2}{Str\+Ne()}} work for wide strings as well.

\subsection*{Container Matchers}

Most S\+T\+L-\/style containers support {\ttfamily ==}, so you can use {\ttfamily Eq(expected\+\_\+container)} or simply {\ttfamily expected\+\_\+container} to match a container exactly. If you want to write the elements in-\/line, match them more flexibly, or get more informative messages, you can use\+:

\tabulinesep=1mm
\begin{longtabu} spread 0pt [c]{*2{|X[-1]}|}
\hline
\rowcolor{\tableheadbgcolor}{\bf {\ttfamily Container\+Eq(container)} }&{\bf The same as {\ttfamily Eq(container)} except that the failure message also includes which elements are in one container but not the other.  }\\\cline{1-2}
\endfirsthead
\hline
\endfoot
\hline
\rowcolor{\tableheadbgcolor}{\bf {\ttfamily Container\+Eq(container)} }&{\bf The same as {\ttfamily Eq(container)} except that the failure message also includes which elements are in one container but not the other.  }\\\cline{1-2}
\endhead
{\ttfamily Contains(e)} &{\ttfamily argument} contains an element that matches {\ttfamily e}, which can be either a value or a matcher. \\\cline{1-2}
{\ttfamily Each(e)} &{\ttfamily argument} is a container where {\itshape every} element matches {\ttfamily e}, which can be either a value or a matcher. \\\cline{1-2}
{\ttfamily Elements\+Are(e0, e1, ..., en)} &{\ttfamily argument} has {\ttfamily n + 1} elements, where the i-\/th element matches {\ttfamily ei}, which can be a value or a matcher. 0 to 10 arguments are allowed. \\\cline{1-2}
{\ttfamily Elements\+Are\+Array(\{ e0, e1, ..., en \})}, {\ttfamily Elements\+Are\+Array(array)}, or {\ttfamily Elements\+Are\+Array(array, count)} &The same as {\ttfamily \hyperlink{namespacetesting_a79cf4ae694bf8231dcf283b325405f27}{Elements\+Are()}} except that the expected element values/matchers come from an initializer list, S\+T\+L-\/style container, or C-\/style array. \\\cline{1-2}
{\ttfamily Is\+Empty()} &{\ttfamily argument} is an empty container ({\ttfamily container.\+empty()}). \\\cline{1-2}
{\ttfamily Pointwise(m, container)} &{\ttfamily argument} contains the same number of elements as in {\ttfamily container}, and for all i, (the i-\/th element in {\ttfamily argument}, the i-\/th element in {\ttfamily container}) match {\ttfamily m}, which is a matcher on 2-\/tuples. E.\+g. {\ttfamily Pointwise(\+Le(), upper\+\_\+bounds)} verifies that each element in {\ttfamily argument} doesn\textquotesingle{}t exceed the corresponding element in {\ttfamily upper\+\_\+bounds}. See more detail below. \\\cline{1-2}
{\ttfamily Size\+Is(m)} &{\ttfamily argument} is a container whose size matches {\ttfamily m}. E.\+g. {\ttfamily Size\+Is(2)} or {\ttfamily Size\+Is(\+Lt(2))}. \\\cline{1-2}
{\ttfamily Unordered\+Elements\+Are(e0, e1, ..., en)} &{\ttfamily argument} has {\ttfamily n + 1} elements, and under some permutation each element matches an {\ttfamily ei} (for a different {\ttfamily i}), which can be a value or a matcher. 0 to 10 arguments are allowed. \\\cline{1-2}
{\ttfamily Unordered\+Elements\+Are\+Array(\{ e0, e1, ..., en \})}, {\ttfamily Unordered\+Elements\+Are\+Array(array)}, or {\ttfamily Unordered\+Elements\+Are\+Array(array, count)} &The same as {\ttfamily \hyperlink{namespacetesting_a8622c12aadfa0e60f7d68683eeb21115}{Unordered\+Elements\+Are()}} except that the expected element values/matchers come from an initializer list, S\+T\+L-\/style container, or C-\/style array. \\\cline{1-2}
{\ttfamily When\+Sorted(m)} &When {\ttfamily argument} is sorted using the {\ttfamily $<$} operator, it matches container matcher {\ttfamily m}. E.\+g. {\ttfamily When\+Sorted(\+Unordered\+Elements\+Are(1, 2, 3))} verifies that {\ttfamily argument} contains elements {\ttfamily 1}, {\ttfamily 2}, and {\ttfamily 3}, ignoring order. \\\cline{1-2}
{\ttfamily When\+Sorted\+By(comparator, m)} &The same as {\ttfamily When\+Sorted(m)}, except that the given comparator instead of {\ttfamily $<$} is used to sort {\ttfamily argument}. E.\+g. {\ttfamily When\+Sorted\+By(std\+::greater$<$int$>$(), Elements\+Are(3, 2, 1))}. \\\cline{1-2}
\end{longtabu}
Notes\+:


\begin{DoxyItemize}
\item These matchers can also match\+:
\begin{DoxyEnumerate}
\item a native array passed by reference (e.\+g. in {\ttfamily Foo(const int (\&a)\mbox{[}5\mbox{]})}), and
\end{DoxyEnumerate}
\begin{DoxyEnumerate}
\item an array passed as a pointer and a count (e.\+g. in {\ttfamily Bar(const T$\ast$ buffer, int len)} -- see Multi-\/argument Matchers).
\end{DoxyEnumerate}
\item The array being matched may be multi-\/dimensional (i.\+e. its elements can be arrays).
\item {\ttfamily m} in {\ttfamily Pointwise(m, ...)} should be a matcher for {\ttfamily \+::testing\+::tuple$<$T, U$>$} where {\ttfamily T} and {\ttfamily U} are the element type of the actual container and the expected container, respectively. For example, to compare two {\ttfamily Foo} containers where {\ttfamily Foo} doesn\textquotesingle{}t support {\ttfamily operator==} but has an {\ttfamily Equals()} method, one might write\+:
\end{DoxyItemize}


\begin{DoxyCode}
1 using ::testing::get;
2 MATCHER(FooEq, "") \{
3   return get<0>(arg).Equals(get<1>(arg));
4 \}
5 ...
6 EXPECT\_THAT(actual\_foos, Pointwise(FooEq(), expected\_foos));
\end{DoxyCode}


\subsection*{Member Matchers}

\tabulinesep=1mm
\begin{longtabu} spread 0pt [c]{*2{|X[-1]}|}
\hline
\rowcolor{\tableheadbgcolor}{\bf {\ttfamily Field(\&class\+::field, m)}}&{\bf {\ttfamily argument.\+field} (or {\ttfamily argument-\/$>$field} when {\ttfamily argument} is a plain pointer) matches matcher {\ttfamily m}, where {\ttfamily argument} is an object of type {\itshape class}.  }\\\cline{1-2}
\endfirsthead
\hline
\endfoot
\hline
\rowcolor{\tableheadbgcolor}{\bf {\ttfamily Field(\&class\+::field, m)}}&{\bf {\ttfamily argument.\+field} (or {\ttfamily argument-\/$>$field} when {\ttfamily argument} is a plain pointer) matches matcher {\ttfamily m}, where {\ttfamily argument} is an object of type {\itshape class}.  }\\\cline{1-2}
\endhead
{\ttfamily Key(e)} &{\ttfamily argument.\+first} matches {\ttfamily e}, which can be either a value or a matcher. E.\+g. {\ttfamily Contains(Key(\+Le(5)))} can verify that a {\ttfamily map} contains a key {\ttfamily $<$= 5}. \\\cline{1-2}
{\ttfamily Pair(m1, m2)} &{\ttfamily argument} is an {\ttfamily std\+::pair} whose {\ttfamily first} field matches {\ttfamily m1} and {\ttfamily second} field matches {\ttfamily m2}. \\\cline{1-2}
{\ttfamily Property(\&class\+::property, m)}&{\ttfamily argument.\+property()} (or {\ttfamily argument-\/$>$property()} when {\ttfamily argument} is a plain pointer) matches matcher {\ttfamily m}, where {\ttfamily argument} is an object of type {\itshape class}. \\\cline{1-2}
\end{longtabu}
\subsection*{Matching the Result of a Function or Functor}

$\vert${\ttfamily Result\+Of(f, m)}$\vert${\ttfamily f(argument)} matches matcher {\ttfamily m}, where {\ttfamily f} is a function or functor.$\vert$ $\vert$\+:-\/-\/-\/-\/-\/-\/-\/-\/-\/-\/-\/-\/---$\vert$\+:-\/-\/-\/-\/-\/-\/-\/-\/-\/-\/-\/-\/-\/-\/-\/-\/-\/-\/-\/-\/-\/-\/-\/-\/-\/-\/-\/-\/-\/-\/-\/-\/-\/-\/-\/-\/-\/-\/-\/-\/-\/-\/-\/-\/-\/-\/-\/-\/-\/-\/-\/-\/-\/-\/-\/-\/-\/-\/-\/-\/-\/-\/-\/-\/-\/-\/---$\vert$

\subsection*{Pointer Matchers}

\tabulinesep=1mm
\begin{longtabu} spread 0pt [c]{*2{|X[-1]}|}
\hline
\rowcolor{\tableheadbgcolor}{\bf {\ttfamily Pointee(m)}}&{\bf {\ttfamily argument} (either a smart pointer or a raw pointer) points to a value that matches matcher {\ttfamily m}.  }\\\cline{1-2}
\endfirsthead
\hline
\endfoot
\hline
\rowcolor{\tableheadbgcolor}{\bf {\ttfamily Pointee(m)}}&{\bf {\ttfamily argument} (either a smart pointer or a raw pointer) points to a value that matches matcher {\ttfamily m}.  }\\\cline{1-2}
\endhead
{\ttfamily When\+Dynamic\+Cast\+To$<$T$>$(m)}&when {\ttfamily argument} is passed through {\ttfamily dynamic\+\_\+cast$<$T$>$()}, it matches matcher {\ttfamily m}. \\\cline{1-2}
\end{longtabu}
\subsection*{Multiargument Matchers}

Technically, all matchers match a {\itshape single} value. A \char`\"{}multi-\/argument\char`\"{} matcher is just one that matches a {\itshape tuple}. The following matchers can be used to match a tuple {\ttfamily (x, y)}\+:

\tabulinesep=1mm
\begin{longtabu} spread 0pt [c]{*2{|X[-1]}|}
\hline
\rowcolor{\tableheadbgcolor}{\bf {\ttfamily \hyperlink{namespacetesting_a0cb8ba7eae844c871eccb29e7c81635f}{Eq()}}}&{\bf {\ttfamily x == y}  }\\\cline{1-2}
\endfirsthead
\hline
\endfoot
\hline
\rowcolor{\tableheadbgcolor}{\bf {\ttfamily \hyperlink{namespacetesting_a0cb8ba7eae844c871eccb29e7c81635f}{Eq()}}}&{\bf {\ttfamily x == y}  }\\\cline{1-2}
\endhead
{\ttfamily \hyperlink{namespacetesting_a42bb19b42d7830b972973a103d5e00f2}{Ge()}}&{\ttfamily x $>$= y} \\\cline{1-2}
{\ttfamily \hyperlink{namespacetesting_a493fc1bafd7b3945ba06ace80e74b0d5}{Gt()}}&{\ttfamily x $>$ y} \\\cline{1-2}
{\ttfamily \hyperlink{namespacetesting_a2e33596921b80a7fdaff3f62bf18a478}{Le()}}&{\ttfamily x $<$= y} \\\cline{1-2}
{\ttfamily \hyperlink{namespacetesting_ad621459957a8bcdd3c256b7940ecbf99}{Lt()}}&{\ttfamily x $<$ y} \\\cline{1-2}
{\ttfamily \hyperlink{namespacetesting_afe42d41d5171234cb9da5da27faeb7e8}{Ne()}}&{\ttfamily x != y} \\\cline{1-2}
\end{longtabu}
You can use the following selectors to pick a subset of the arguments (or reorder them) to participate in the matching\+:

\tabulinesep=1mm
\begin{longtabu} spread 0pt [c]{*2{|X[-1]}|}
\hline
\rowcolor{\tableheadbgcolor}{\bf {\ttfamily All\+Args(m)}}&{\bf Equivalent to {\ttfamily m}. Useful as syntactic sugar in {\ttfamily .With(\+All\+Args(m))}.  }\\\cline{1-2}
\endfirsthead
\hline
\endfoot
\hline
\rowcolor{\tableheadbgcolor}{\bf {\ttfamily All\+Args(m)}}&{\bf Equivalent to {\ttfamily m}. Useful as syntactic sugar in {\ttfamily .With(\+All\+Args(m))}.  }\\\cline{1-2}
\endhead
{\ttfamily Args$<$N1, N2, ..., Nk$>$(m)}&The tuple of the {\ttfamily k} selected (using 0-\/based indices) arguments matches {\ttfamily m}, e.\+g. {\ttfamily Args$<$1, 2$>$(\hyperlink{namespacetesting_a0cb8ba7eae844c871eccb29e7c81635f}{Eq()})}. \\\cline{1-2}
\end{longtabu}
\subsection*{Composite Matchers}

You can make a matcher from one or more other matchers\+:

\tabulinesep=1mm
\begin{longtabu} spread 0pt [c]{*2{|X[-1]}|}
\hline
\rowcolor{\tableheadbgcolor}{\bf {\ttfamily All\+Of(m1, m2, ..., mn)}}&{\bf {\ttfamily argument} matches all of the matchers {\ttfamily m1} to {\ttfamily mn}.  }\\\cline{1-2}
\endfirsthead
\hline
\endfoot
\hline
\rowcolor{\tableheadbgcolor}{\bf {\ttfamily All\+Of(m1, m2, ..., mn)}}&{\bf {\ttfamily argument} matches all of the matchers {\ttfamily m1} to {\ttfamily mn}.  }\\\cline{1-2}
\endhead
{\ttfamily Any\+Of(m1, m2, ..., mn)}&{\ttfamily argument} matches at least one of the matchers {\ttfamily m1} to {\ttfamily mn}. \\\cline{1-2}
{\ttfamily Not(m)} &{\ttfamily argument} doesn\textquotesingle{}t match matcher {\ttfamily m}. \\\cline{1-2}
\end{longtabu}
\subsection*{Adapters for Matchers}

\tabulinesep=1mm
\begin{longtabu} spread 0pt [c]{*2{|X[-1]}|}
\hline
\rowcolor{\tableheadbgcolor}{\bf {\ttfamily Matcher\+Cast$<$T$>$(m)}}&{\bf casts matcher {\ttfamily m} to type {\ttfamily Matcher$<$T$>$}.  }\\\cline{1-2}
\endfirsthead
\hline
\endfoot
\hline
\rowcolor{\tableheadbgcolor}{\bf {\ttfamily Matcher\+Cast$<$T$>$(m)}}&{\bf casts matcher {\ttfamily m} to type {\ttfamily Matcher$<$T$>$}.  }\\\cline{1-2}
\endhead
{\ttfamily Safe\+Matcher\+Cast$<$T$>$(m)}&\href{CookBook.md#casting-matchers}{\tt safely casts} matcher {\ttfamily m} to type {\ttfamily Matcher$<$T$>$}. \\\cline{1-2}
{\ttfamily Truly(predicate)} &{\ttfamily predicate(argument)} returns something considered by C++ to be true, where {\ttfamily predicate} is a function or functor. \\\cline{1-2}
\end{longtabu}
\subsection*{Matchers as Predicates}

\tabulinesep=1mm
\begin{longtabu} spread 0pt [c]{*2{|X[-1]}|}
\hline
\rowcolor{\tableheadbgcolor}{\bf {\ttfamily Matches(m)(value)}}&{\bf evaluates to {\ttfamily true} if {\ttfamily value} matches {\ttfamily m}. You can use {\ttfamily Matches(m)} alone as a unary functor.  }\\\cline{1-2}
\endfirsthead
\hline
\endfoot
\hline
\rowcolor{\tableheadbgcolor}{\bf {\ttfamily Matches(m)(value)}}&{\bf evaluates to {\ttfamily true} if {\ttfamily value} matches {\ttfamily m}. You can use {\ttfamily Matches(m)} alone as a unary functor.  }\\\cline{1-2}
\endhead
{\ttfamily Explain\+Match\+Result(m, value, result\+\_\+listener)}&evaluates to {\ttfamily true} if {\ttfamily value} matches {\ttfamily m}, explaining the result to {\ttfamily result\+\_\+listener}. \\\cline{1-2}
{\ttfamily Value(value, m)} &evaluates to {\ttfamily true} if {\ttfamily value} matches {\ttfamily m}. \\\cline{1-2}
\end{longtabu}
\subsection*{Defining Matchers}

\tabulinesep=1mm
\begin{longtabu} spread 0pt [c]{*2{|X[-1]}|}
\hline
\rowcolor{\tableheadbgcolor}{\bf {\ttfamily M\+A\+T\+C\+H\+ER(Is\+Even, \char`\"{}\char`\"{}) \{ return (arg \% 2) == 0; \}} }&{\bf Defines a matcher {\ttfamily Is\+Even()} to match an even number.  }\\\cline{1-2}
\endfirsthead
\hline
\endfoot
\hline
\rowcolor{\tableheadbgcolor}{\bf {\ttfamily M\+A\+T\+C\+H\+ER(Is\+Even, \char`\"{}\char`\"{}) \{ return (arg \% 2) == 0; \}} }&{\bf Defines a matcher {\ttfamily Is\+Even()} to match an even number.  }\\\cline{1-2}
\endhead
{\ttfamily M\+A\+T\+C\+H\+E\+R\+\_\+P(Is\+Divisible\+By, n, \char`\"{}\char`\"{}) \{ $\ast$result\+\_\+listener $<$$<$ \char`\"{}where the remainder is \char`\"{} $<$$<$ (arg \% n); return (arg \% n) == 0; \}} &Defines a macher {\ttfamily Is\+Divisible\+By(n)} to match a number divisible by {\ttfamily n}. \\\cline{1-2}
{\ttfamily M\+A\+T\+C\+H\+E\+R\+\_\+\+P2(Is\+Between, a, b, std\+::string(negation ? \char`\"{}isn\textquotesingle{}t\char`\"{} \+: \char`\"{}is\char`\"{}) + \char`\"{} between \char`\"{} + Print\+To\+String(a) + \char`\"{} and \char`\"{} + Print\+To\+String(b)) \{ return a $<$= arg \&\& arg $<$= b; \}} &Defines a matcher {\ttfamily Is\+Between(a, b)} to match a value in the range \mbox{[}{\ttfamily a}, {\ttfamily b}\mbox{]}. \\\cline{1-2}
\end{longtabu}
{\bfseries Notes\+:}


\begin{DoxyEnumerate}
\item The {\ttfamily M\+A\+T\+C\+H\+E\+R$\ast$} macros cannot be used inside a function or class.
\end{DoxyEnumerate}
\begin{DoxyEnumerate}
\item The matcher body must be {\itshape purely functional} (i.\+e. it cannot have any side effect, and the result must not depend on anything other than the value being matched and the matcher parameters).
\end{DoxyEnumerate}
\begin{DoxyEnumerate}
\item You can use {\ttfamily Print\+To\+String(x)} to convert a value {\ttfamily x} of any type to a string.
\end{DoxyEnumerate}

\subsection*{Matchers as Test Assertions}

\tabulinesep=1mm
\begin{longtabu} spread 0pt [c]{*2{|X[-1]}|}
\hline
\rowcolor{\tableheadbgcolor}{\bf {\ttfamily \hyperlink{gmock-matchers_8h_a41d888579850c16583baea33ee8d057e}{A\+S\+S\+E\+R\+T\+\_\+\+T\+H\+A\+T(expression, m)}}}&{\bf Generates a \href{../../googletest/docs/Primer.md#assertions}{\tt fatal failure} if the value of {\ttfamily expression} doesn\textquotesingle{}t match matcher {\ttfamily m}.  }\\\cline{1-2}
\endfirsthead
\hline
\endfoot
\hline
\rowcolor{\tableheadbgcolor}{\bf {\ttfamily \hyperlink{gmock-matchers_8h_a41d888579850c16583baea33ee8d057e}{A\+S\+S\+E\+R\+T\+\_\+\+T\+H\+A\+T(expression, m)}}}&{\bf Generates a \href{../../googletest/docs/Primer.md#assertions}{\tt fatal failure} if the value of {\ttfamily expression} doesn\textquotesingle{}t match matcher {\ttfamily m}.  }\\\cline{1-2}
\endhead
{\ttfamily \hyperlink{gmock-matchers_8h_ac31e206123aa702e1152bb2735b31409}{E\+X\+P\+E\+C\+T\+\_\+\+T\+H\+A\+T(expression, m)}}&Generates a non-\/fatal failure if the value of {\ttfamily expression} doesn\textquotesingle{}t match matcher {\ttfamily m}. \\\cline{1-2}
\end{longtabu}
\section*{Actions}

{\bfseries Actions} specify what a mock function should do when invoked.

\subsection*{Returning a Value}

\tabulinesep=1mm
\begin{longtabu} spread 0pt [c]{*2{|X[-1]}|}
\hline
\rowcolor{\tableheadbgcolor}{\bf {\ttfamily \hyperlink{namespacetesting_af6d1c13e9376c77671e37545cd84359c}{Return()}}}&{\bf Return from a {\ttfamily void} mock function.  }\\\cline{1-2}
\endfirsthead
\hline
\endfoot
\hline
\rowcolor{\tableheadbgcolor}{\bf {\ttfamily \hyperlink{namespacetesting_af6d1c13e9376c77671e37545cd84359c}{Return()}}}&{\bf Return from a {\ttfamily void} mock function.  }\\\cline{1-2}
\endhead
{\ttfamily Return(value)}&Return {\ttfamily value}. If the type of {\ttfamily value} is different to the mock function\textquotesingle{}s return type, {\ttfamily value} is converted to the latter type {\itshape at the time the expectation is set}, not when the action is executed. \\\cline{1-2}
{\ttfamily Return\+Arg$<$N$>$()}&Return the {\ttfamily N}-\/th (0-\/based) argument. \\\cline{1-2}
{\ttfamily Return\+New$<$T$>$(a1, ..., ak)}&Return {\ttfamily new T(a1, ..., ak)}; a different object is created each time. \\\cline{1-2}
{\ttfamily \hyperlink{namespacetesting_aa0331596e269114da101f810d3a1b88b}{Return\+Null()}}&Return a null pointer. \\\cline{1-2}
{\ttfamily Return\+Pointee(ptr)}&Return the value pointed to by {\ttfamily ptr}. \\\cline{1-2}
{\ttfamily Return\+Ref(variable)}&Return a reference to {\ttfamily variable}. \\\cline{1-2}
{\ttfamily Return\+Ref\+Of\+Copy(value)}&Return a reference to a copy of {\ttfamily value}; the copy lives as long as the action. \\\cline{1-2}
\end{longtabu}
\subsection*{Side Effects}

\tabulinesep=1mm
\begin{longtabu} spread 0pt [c]{*2{|X[-1]}|}
\hline
\rowcolor{\tableheadbgcolor}{\bf {\ttfamily Assign(\&variable, value)}}&{\bf Assign {\ttfamily value} to variable.  }\\\cline{1-2}
\endfirsthead
\hline
\endfoot
\hline
\rowcolor{\tableheadbgcolor}{\bf {\ttfamily Assign(\&variable, value)}}&{\bf Assign {\ttfamily value} to variable.  }\\\cline{1-2}
\endhead
{\ttfamily Delete\+Arg$<$N$>$()} &Delete the {\ttfamily N}-\/th (0-\/based) argument, which must be a pointer. \\\cline{1-2}
{\ttfamily Save\+Arg$<$N$>$(pointer)} &Save the {\ttfamily N}-\/th (0-\/based) argument to {\ttfamily $\ast$pointer}. \\\cline{1-2}
{\ttfamily Save\+Arg\+Pointee$<$N$>$(pointer)} &Save the value pointed to by the {\ttfamily N}-\/th (0-\/based) argument to {\ttfamily $\ast$pointer}. \\\cline{1-2}
{\ttfamily Set\+Arg\+Referee$<$N$>$(value)} &Assign value to the variable referenced by the {\ttfamily N}-\/th (0-\/based) argument. \\\cline{1-2}
{\ttfamily Set\+Arg\+Pointee$<$N$>$(value)} &Assign {\ttfamily value} to the variable pointed by the {\ttfamily N}-\/th (0-\/based) argument. \\\cline{1-2}
{\ttfamily Set\+Argument\+Pointee$<$N$>$(value)}&Same as {\ttfamily Set\+Arg\+Pointee$<$N$>$(value)}. Deprecated. Will be removed in v1.\+7.\+0. \\\cline{1-2}
{\ttfamily Set\+Array\+Argument$<$N$>$(first, last)}&Copies the elements in source range \mbox{[}{\ttfamily first}, {\ttfamily last}) to the array pointed to by the {\ttfamily N}-\/th (0-\/based) argument, which can be either a pointer or an iterator. The action does not take ownership of the elements in the source range. \\\cline{1-2}
{\ttfamily Set\+Errno\+And\+Return(error, value)}&Set {\ttfamily errno} to {\ttfamily error} and return {\ttfamily value}. \\\cline{1-2}
{\ttfamily Throw(exception)} &Throws the given exception, which can be any copyable value. Available since v1.\+1.\+0. \\\cline{1-2}
\end{longtabu}
\subsection*{Using a Function or a Functor as an Action}

\tabulinesep=1mm
\begin{longtabu} spread 0pt [c]{*2{|X[-1]}|}
\hline
\rowcolor{\tableheadbgcolor}{\bf {\ttfamily Invoke(f)}}&{\bf Invoke {\ttfamily f} with the arguments passed to the mock function, where {\ttfamily f} can be a global/static function or a functor.  }\\\cline{1-2}
\endfirsthead
\hline
\endfoot
\hline
\rowcolor{\tableheadbgcolor}{\bf {\ttfamily Invoke(f)}}&{\bf Invoke {\ttfamily f} with the arguments passed to the mock function, where {\ttfamily f} can be a global/static function or a functor.  }\\\cline{1-2}
\endhead
{\ttfamily Invoke(object\+\_\+pointer, \&class\+::method)}&Invoke the \{method on the object with the arguments passed to the mock function. \\\cline{1-2}
{\ttfamily Invoke\+Without\+Args(f)}&Invoke {\ttfamily f}, which can be a global/static function or a functor. {\ttfamily f} must take no arguments. \\\cline{1-2}
{\ttfamily Invoke\+Without\+Args(object\+\_\+pointer, \&class\+::method)}&Invoke the method on the object, which takes no arguments. \\\cline{1-2}
{\ttfamily Invoke\+Argument$<$N$>$(arg1, arg2, ..., argk)}&Invoke the mock function\textquotesingle{}s {\ttfamily N}-\/th (0-\/based) argument, which must be a function or a functor, with the {\ttfamily k} arguments. \\\cline{1-2}
\end{longtabu}
The return value of the invoked function is used as the return value of the action.

When defining a function or functor to be used with {\ttfamily Invoke$\ast$()}, you can declare any unused parameters as {\ttfamily Unused}\+: 
\begin{DoxyCode}
1 double Distance(Unused, double x, double y) \{ return sqrt(x*x + y*y); \}
2 ...
3 EXPECT\_CALL(mock, Foo("Hi", \_, \_)).WillOnce(Invoke(Distance));
\end{DoxyCode}


In {\ttfamily Invoke\+Argument$<$N$>$(...)}, if an argument needs to be passed by reference, wrap it inside {\ttfamily \hyperlink{namespacetesting_aaee6d42dcd69de6e7a1459c5c71222c3}{By\+Ref()}}. For example, 
\begin{DoxyCode}
1 InvokeArgument<2>(5, string("Hi"), ByRef(foo))
\end{DoxyCode}
 calls the mock function\textquotesingle{}s \#2 argument, passing to it {\ttfamily 5} and {\ttfamily string(\char`\"{}\+Hi\char`\"{})} by value, and {\ttfamily foo} by reference.

\subsection*{Default Action}

$\vert${\ttfamily \hyperlink{namespacetesting_ae041df61ff61ccb9753ba15b4309e1a6}{Do\+Default()}}$\vert$\+Do the default action (specified by {\ttfamily \hyperlink{gmock-spec-builders_8h_a5b12ae6cf84f0a544ca811b380c37334}{O\+N\+\_\+\+C\+A\+L\+L()}} or the built-\/in one).$\vert$ $\vert$\+:-\/-\/-\/-\/-\/-\/-\/-\/-\/---$\vert$\+:-\/-\/-\/-\/-\/-\/-\/-\/-\/-\/-\/-\/-\/-\/-\/-\/-\/-\/-\/-\/-\/-\/-\/-\/-\/-\/-\/-\/-\/-\/-\/-\/-\/-\/-\/-\/-\/-\/-\/-\/-\/-\/-\/-\/-\/-\/-\/-\/-\/-\/-\/-\/-\/-\/-\/-\/-\/-\/-\/-\/-\/-\/-\/-\/-\/---$\vert$

{\bfseries Note\+:} due to technical reasons, {\ttfamily \hyperlink{namespacetesting_ae041df61ff61ccb9753ba15b4309e1a6}{Do\+Default()}} cannot be used inside a composite action -\/ trying to do so will result in a run-\/time error.

\subsection*{Composite Actions}

\tabulinesep=1mm
\begin{longtabu} spread 0pt [c]{*2{|X[-1]}|}
\hline
\rowcolor{\tableheadbgcolor}{\bf {\ttfamily Do\+All(a1, a2, ..., an)}}&{\bf Do all actions {\ttfamily a1} to {\ttfamily an} and return the result of {\ttfamily an} in each invocation. The first {\ttfamily n -\/ 1} sub-\/actions must return void.  }\\\cline{1-2}
\endfirsthead
\hline
\endfoot
\hline
\rowcolor{\tableheadbgcolor}{\bf {\ttfamily Do\+All(a1, a2, ..., an)}}&{\bf Do all actions {\ttfamily a1} to {\ttfamily an} and return the result of {\ttfamily an} in each invocation. The first {\ttfamily n -\/ 1} sub-\/actions must return void.  }\\\cline{1-2}
\endhead
{\ttfamily Ignore\+Result(a)} &Perform action {\ttfamily a} and ignore its result. {\ttfamily a} must not return void. \\\cline{1-2}
{\ttfamily With\+Arg$<$N$>$(a)} &Pass the {\ttfamily N}-\/th (0-\/based) argument of the mock function to action {\ttfamily a} and perform it. \\\cline{1-2}
{\ttfamily With\+Args$<$N1, N2, ..., Nk$>$(a)}&Pass the selected (0-\/based) arguments of the mock function to action {\ttfamily a} and perform it. \\\cline{1-2}
{\ttfamily Without\+Args(a)} &Perform action {\ttfamily a} without any arguments. \\\cline{1-2}
\end{longtabu}
\subsection*{Defining Actions}

\tabulinesep=1mm
\begin{longtabu} spread 0pt [c]{*2{|X[-1]}|}
\hline
\rowcolor{\tableheadbgcolor}{\bf {\ttfamily \hyperlink{gmock-generated-actions_8h_a7af7137aa4871df4235881af377205fe}{A\+C\+T\+I\+O\+N(\+Sum)} \{ return arg0 + arg1; \}} }&{\bf Defines an action {\ttfamily Sum()} to return the sum of the mock function\textquotesingle{}s argument \#0 and \#1.  }\\\cline{1-2}
\endfirsthead
\hline
\endfoot
\hline
\rowcolor{\tableheadbgcolor}{\bf {\ttfamily \hyperlink{gmock-generated-actions_8h_a7af7137aa4871df4235881af377205fe}{A\+C\+T\+I\+O\+N(\+Sum)} \{ return arg0 + arg1; \}} }&{\bf Defines an action {\ttfamily Sum()} to return the sum of the mock function\textquotesingle{}s argument \#0 and \#1.  }\\\cline{1-2}
\endhead
{\ttfamily \hyperlink{gmock-generated-actions_8h_a8ee9766f611f068271ca37a90c0e5960}{A\+C\+T\+I\+O\+N\+\_\+\+P(\+Plus, n)} \{ return arg0 + n; \}} &Defines an action {\ttfamily Plus(n)} to return the sum of the mock function\textquotesingle{}s argument \#0 and {\ttfamily n}. \\\cline{1-2}
{\ttfamily A\+C\+T\+I\+O\+N\+\_\+\+Pk(Foo, p1, ..., pk) \{ statements; \}} &Defines a parameterized action {\ttfamily Foo(p1, ..., pk)} to execute the given {\ttfamily statements}. \\\cline{1-2}
\end{longtabu}
The {\ttfamily A\+C\+T\+I\+O\+N$\ast$} macros cannot be used inside a function or class.

\section*{Cardinalities}

These are used in {\ttfamily Times()} to specify how many times a mock function will be called\+:

\tabulinesep=1mm
\begin{longtabu} spread 0pt [c]{*2{|X[-1]}|}
\hline
\rowcolor{\tableheadbgcolor}{\bf {\ttfamily \hyperlink{namespacetesting_aa1f8a6371097e1e9b8d6866020f35252}{Any\+Number()}}}&{\bf The function can be called any number of times.  }\\\cline{1-2}
\endfirsthead
\hline
\endfoot
\hline
\rowcolor{\tableheadbgcolor}{\bf {\ttfamily \hyperlink{namespacetesting_aa1f8a6371097e1e9b8d6866020f35252}{Any\+Number()}}}&{\bf The function can be called any number of times.  }\\\cline{1-2}
\endhead
{\ttfamily At\+Least(n)} &The call is expected at least {\ttfamily n} times. \\\cline{1-2}
{\ttfamily At\+Most(n)} &The call is expected at most {\ttfamily n} times. \\\cline{1-2}
{\ttfamily Between(m, n)}&The call is expected between {\ttfamily m} and {\ttfamily n} (inclusive) times. \\\cline{1-2}
{\ttfamily Exactly(n) or n}&The call is expected exactly {\ttfamily n} times. In particular, the call should never happen when {\ttfamily n} is 0. \\\cline{1-2}
\end{longtabu}
\section*{Expectation Order}

By default, the expectations can be matched in {\itshape any} order. If some or all expectations must be matched in a given order, there are two ways to specify it. They can be used either independently or together.

\subsection*{The After Clause}


\begin{DoxyCode}
1 using ::testing::Expectation;
2 ...
3 Expectation init\_x = EXPECT\_CALL(foo, InitX());
4 Expectation init\_y = EXPECT\_CALL(foo, InitY());
5 EXPECT\_CALL(foo, Bar())
6     .After(init\_x, init\_y);
\end{DoxyCode}
 says that {\ttfamily Bar()} can be called only after both {\ttfamily Init\+X()} and {\ttfamily Init\+Y()} have been called.

If you don\textquotesingle{}t know how many pre-\/requisites an expectation has when you write it, you can use an {\ttfamily Expectation\+Set} to collect them\+:


\begin{DoxyCode}
1 using ::testing::ExpectationSet;
2 ...
3 ExpectationSet all\_inits;
4 for (int i = 0; i < element\_count; i++) \{
5   all\_inits += EXPECT\_CALL(foo, InitElement(i));
6 \}
7 EXPECT\_CALL(foo, Bar())
8     .After(all\_inits);
\end{DoxyCode}
 says that {\ttfamily Bar()} can be called only after all elements have been initialized (but we don\textquotesingle{}t care about which elements get initialized before the others).

Modifying an {\ttfamily Expectation\+Set} after using it in an {\ttfamily .After()} doesn\textquotesingle{}t affect the meaning of the {\ttfamily .After()}.

\subsection*{Sequences}

When you have a long chain of sequential expectations, it\textquotesingle{}s easier to specify the order using {\bfseries sequences}, which don\textquotesingle{}t require you to given each expectation in the chain a different name. {\itshape All expected~\newline
 calls} in the same sequence must occur in the order they are specified.


\begin{DoxyCode}
1 using ::testing::Sequence;
2 Sequence s1, s2;
3 ...
4 EXPECT\_CALL(foo, Reset())
5     .InSequence(s1, s2)
6     .WillOnce(Return(true));
7 EXPECT\_CALL(foo, GetSize())
8     .InSequence(s1)
9     .WillOnce(Return(1));
10 EXPECT\_CALL(foo, Describe(A<const char*>()))
11     .InSequence(s2)
12     .WillOnce(Return("dummy"));
\end{DoxyCode}
 says that {\ttfamily Reset()} must be called before {\itshape both} {\ttfamily Get\+Size()} {\itshape and} {\ttfamily \hyperlink{namespacetesting_1_1gmock__matchers__test_a36ae44e4f6c8e9fe3fe9162132503717}{Describe()}}, and the latter two can occur in any order.

To put many expectations in a sequence conveniently\+: 
\begin{DoxyCode}
1 using ::testing::InSequence;
2 \{
3   InSequence dummy;
4 
5   EXPECT\_CALL(...)...;
6   EXPECT\_CALL(...)...;
7   ...
8   EXPECT\_CALL(...)...;
9 \}
\end{DoxyCode}
 says that all expected calls in the scope of {\ttfamily dummy} must occur in strict order. The name {\ttfamily dummy} is irrelevant.)

\section*{Verifying and Resetting a \hyperlink{classMock}{Mock}}

Google \hyperlink{classMock}{Mock} will verify the expectations on a mock object when it is destructed, or you can do it earlier\+: 
\begin{DoxyCode}
1 using ::testing::Mock;
2 ...
3 // Verifies and removes the expectations on mock\_obj;
4 // returns true iff successful.
5 Mock::VerifyAndClearExpectations(&mock\_obj);
6 ...
7 // Verifies and removes the expectations on mock\_obj;
8 // also removes the default actions set by ON\_CALL();
9 // returns true iff successful.
10 Mock::VerifyAndClear(&mock\_obj);
\end{DoxyCode}


You can also tell Google \hyperlink{classMock}{Mock} that a mock object can be leaked and doesn\textquotesingle{}t need to be verified\+: 
\begin{DoxyCode}
1 Mock::AllowLeak(&mock\_obj);
\end{DoxyCode}


\section*{\hyperlink{classMock}{Mock} Classes}

Google \hyperlink{classMock}{Mock} defines a convenient mock class template 
\begin{DoxyCode}
1 class MockFunction<R(A1, ..., An)> \{
2  public:
3   MOCK\_METHODn(Call, R(A1, ..., An));
4 \};
\end{DoxyCode}
 See this \href{CookBook.md#using-check-points}{\tt recipe} for one application of it.

\section*{Flags}

\tabulinesep=1mm
\begin{longtabu} spread 0pt [c]{*2{|X[-1]}|}
\hline
\rowcolor{\tableheadbgcolor}{\bf {\ttfamily -\/-\/gmock\+\_\+catch\+\_\+leaked\+\_\+mocks=0} }&{\bf Don\textquotesingle{}t report leaked mock objects as failures.  }\\\cline{1-2}
\endfirsthead
\hline
\endfoot
\hline
\rowcolor{\tableheadbgcolor}{\bf {\ttfamily -\/-\/gmock\+\_\+catch\+\_\+leaked\+\_\+mocks=0} }&{\bf Don\textquotesingle{}t report leaked mock objects as failures.  }\\\cline{1-2}
\endhead
{\ttfamily -\/-\/gmock\+\_\+verbose=L\+E\+V\+EL} &Sets the default verbosity level ({\ttfamily info}, {\ttfamily warning}, or {\ttfamily error}) of Google \hyperlink{classMock}{Mock} messages. \\\cline{1-2}
\end{longtabu}
